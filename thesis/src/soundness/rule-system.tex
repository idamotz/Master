\section{Soundness of the Rule System}
In this section, we summarise the lemmas proven in Section~\vref{sec:soundness-of-the-rules} and Appendix~\vref{cha:remaining-soundness-proofs} into three theorems that show that the rule system is sound \textemdash{} given a sound plan and an operation, if the operation is applied, then it causes no paradoxes, and the resulting model is well-formed. We base this on the definition of well-formedness for interval-based feature models listed in well-formedness requirements \wf{1\textendash 9}. Furthermore, the rules are modular, meaning that they operate within limited scopes.

In Section~\ref{sec:soundness-of-the-rules} and Appendix~\ref{cha:remaining-soundness-proofs}, we have shown that each of the analysis rules operates within the temporal and spatial scopes of an operation. Based on the lemmas\footnote{See lemmas \ref{lemma:add-feature-scope}, \ref{lemma:move-feature-scope}, \ref{lemma:add-group-scope}, \ref{lemma:remove-feature-scope}, \ref{lemma:remove-group-scope}, \ref{lemma:move-group-scope}, \ref{lemma:change-feature-variation-type-scope}, \ref{lemma:change-group-variation-type-scope}, \ref{lemma:change-feature-name-scope}}, we prove the following theorem.
\\
\begin{theorem}
  The rule system supports local modification and verification of interval-based feature models.
  \label{theorem:scope}
\end{theorem}

We have shown that each of the analysis rules preserves well-formedness of the interval-based feature model (Section~\ref{sec:soundness-of-the-rules} and Appendix~\ref{cha:remaining-soundness-proofs}). The following theorem follows directly from lemmas\footnote{See lemmas \ref{lemma:add-feature-well-formed}, \ref{lemma:move-feature-well-formed}, \ref{lemma:add-group-well-formed}, \ref{lemma:remove-feature-well-formed}, \ref{lemma:remove-group-well-formed}, \ref{lemma:move-group-well-formed}, \ref{lemma:change-feature-variation-type-well-formed}, \ref{lemma:change-group-variation-type-well-formed}, \ref{lemma:change-feature-name-well-formed}}.
\\
\begin{theorem}
  The rule system for local modification of interval-based feature models is well-formed.
  \label{theorem:well-formed}
\end{theorem}

In Section~\vref{sec:soundness-of-the-rules} and Appendix~\ref{cha:remaining-soundness-proofs}, we have shown that each of the analysis rules behaves according to the semantics of the operation in question.\footnote{See lemmas \ref{lemma:add-feature-mod}, \ref{lemma:move-feature-mod}, \ref{lemma:add-group-mod}, \ref{lemma:remove-feature-mod}, \ref{lemma:remove-group-mod}, \ref{lemma:move-group-mod}, \ref{lemma:change-feature-variation-type-mod}, \ref{lemma:change-group-variation-type-mod}, \ref{lemma:change-feature-name-mod}} Using the lemmas proven there, we conclude that this holds for the rule system.
\\
\begin{theorem}
  The rule system for local modification of interval-based feature models modifies the model correctly.
  \label{theorem:modification}
\end{theorem}

The three theorems given here show that the rule system is sound. If applied to a sound interval-based feature model and an operation, they will update the model correctly if the operation does not cause a paradox (Theorem~\ref{theorem:modification}), and the resulting model will be well-formed (Theorem~\ref{theorem:well-formed}). The rules will only check the parts of the plan that may be structurally violated due to the change (Theorem~\ref{theorem:scope}).
