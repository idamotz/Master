
\section{Soundness of the Change Feature Variation Type Rule} 
\label{sec:soundness-of-the-change-feature-variation-type-rule}

See Figure~\vref{rule:change-feature-varation-type} for the \rulefont{Change-Feature-Variation-Type} rule. Let 
\begin{equation*}
   \begin{split}
      \textbf{changeFeatureVariationType}(\var{featureID}, \var{type}) \text{ at } t_n \shove \\
      (\names, \features, \groups)
   \end{split}
\end{equation*}
be the initial state, and
\[
   (\names', \features', \groups')
\]
be the result state after applying the \rulefont{Change-Feature-Variation-Type} rule. Recall that this operation changes the type of the feature with ID $\var{featureID}$ to $\var{type}$.

\paragraph{Modularity}
Recall that the temporal scope of the \rulefont{Change-Feature-Variation-Type} rule is $\interval{t_n}{t_k}$ (Section~\vref{sec:scope}), where $t_k$ is the time point at which the type is originally planned to be changed or the feature is removed. In the rule, this scope is identified by 
   \[
      \containing{F_t}{t_n} = \set{\interval{t_{t_1}}{t_{t_2}}}
   \]
   Here, the time point $t_n$ for changing the feature type is looked up in the feature's types map's set of interval keys, and the expected result is $\set{\interval{t_{t_1}}{t_{t_2}}}$. This means that there is a mapping $\intervalmapping{t_{t_1}}{t_{t_2}}{\var{oldType}}$ in $F_t$, with $\var{oldType}$ being the type of the feature at time $t_n$, and this stops being the case at $t_{t_2}$. Thus the temporal scope of this operation is $\interval{t_n}{t_{t_2}}$. The only interval looked up or assigned to in the rule is $\interval{t_n}{t_{t_2}}$, so the rule operates strictly within the temporal scope of the operation.

   The spatial scope for this operation is the feature itself and its parent group. Since the feature may move during the temporal scope, there may be several parent groups to consider. These groups and their types are looked up in the premise
\[
   \begin{split}
      \forall \interval{t_{p_1}}{t_{p_2}} \in \overlapping{F_p}{t_n}{t_{t_2}}  \\ % for all parent intervals overlapping the temporal scope
      \forall p \in \lookup{F_p}{\interval{t_{p_1}}{t_{p_2}}}  \\% for all parent groups (always exactly one) in the parent interval
      \forall t \in \, \var{getTypes}\left(\lookup{\groups}{p}, \clamp{\interval{t_{p_1}}{t_{p_2}}}{t_n}{t_{t_2}}\right)  \\
          \big(\var{compatibleTypes}(t, \var{type})\big) 
    \end{split}
\]

Otherwise, the only feature or group looked up or assigned to in the rule is $\lookup{features}{\var{featureID}}$, so the rule stays within the spatial scope.

Based on the above proof, we conclude with the following lemma:
\\

\begin{lemma}
   The \rulefont{Change-Feature-Variation-Type} rule operates strictly within the temporal and spatial scopes of the \textbf{changeFeatureVariationType} operation.
   \label{lemma:change-feature-variation-type-scope}
\end{lemma}

\paragraph{Preserving well-formedness}
Due to the premise $\var{featureID} \neq RootID$, the feature is not the root, so \wf{1\textendash 2} hold trivially. The modification to $F_t$
\[
   \lookup{\var{clampInterval}(F_t, t_n)}{\interval{t_n}{t_{t_2}}} \assign \var{type}
\]
ensures that the feature's original type stops at $t_n$ and the new one lasts for the duration of the temporal scope $\interval{t_n}{t_{t_2}}$. Since the feature has exactly one type during the temporal scope, and no other modifications are made to the feature, \wf{3} is preserved. Because of this, and since the $\groups{}$ map is also left unchanged, \wf{4\textendash 6} and \wf{8\textendash 9} hold. 

As discussed in the \textbf{Scope} paragraph, the premise 
\[
   \begin{split}
      \forall \interval{t_{p_1}}{t_{p_2}} \in \overlapping{F_p}{t_n}{t_{t_2}}  \\ % for all parent intervals overlapping the temporal scope
      \forall p \in \lookup{F_p}{\interval{t_{p_1}}{t_{p_2}}}  \\% for all parent groups (always exactly one) in the parent interval
      \forall t \in \, \var{getTypes}\left(\lookup{\groups}{p}, \clamp{\interval{t_{p_1}}{t_{p_2}}}{t_n}{t_{t_2}}\right)  \\
          \big(\var{compatibleTypes}(t, \var{type})\big) 
    \end{split}
\]
looks up all parent mappings overlapping the temporal scope ($\intervalmapping{t_{p_1}}{t_{p_2}}{p}$), finds the types each parent group has during the scope and \emph{while} it is the parent of the feature, and verifies that those types are compatible. Thus \wf{7} is preserved.

We conclude that the \rulefont{Change-Feature-Variation-Type} rule preserves well-formedness for the interval-based feature model, according to well-formedness rules \wf{1-9}. 
\\

\begin{lemma}
   The \rulefont{Change-Feature-Variation-Type} rule preserves well-formedness of the interval-based feature model. 
   \label{lemma:change-feature-variation-type-well-formed}
\end{lemma}

\paragraph{Correctness of model modification}
The expected result of applying the rule is that $\lookup{\features'}{\var{featureID}} = \featurevar{F'}$ has the type $\var{type}$ during the temporal scope $\interval{t_n}{t_{t_2}}$. Indeed, due to the semantics of $\var{clampInterval}$ and assignment, for any time point $t_i$ such that $t_n \leq t_i < t_{t_2}$,
\[
   \lookup{F'_t}{t_i} = \set{\var{type}}
\]
Since no other part of the interval-based feature model is altered, the rule performs as desired.
\\

\begin{lemma}
   The \rulefont{Change-Feature-Variation-Type} rule updates the interval-based feature model according to the semantics of the \textbf{changeFeatureVariationType} operation.
   \label{lemma:change-feature-variation-type-mod}
\end{lemma}
