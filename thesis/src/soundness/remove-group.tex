\section{Soundness of the Remove Group Rule}
\label{sec:soundness-of-the-remove-group-rule}
This proof is analogous to the one for the \rulefont{Remove-Feature} rule.
See Figure~\vref{rule:remove-group} for the \rulefont{Remove-Group} rule. Let
\[
      \textbf{removeGroup}(\var{groupID}) \text{ at } t_n \shove (\names, \features, \groups)
\]
be the initial state, and
\[
   (\names', \features', \groups')
\]
be the result state after applying the \rulefont{Remove-Group} rule. Recall that this operation removes the group with ID $\var{groupID}$ from the interval-based feature model $(\names, \features, \groups)$ at $t_n$. Furthermore, let $\lookup{\groups}{\var{groupID}} = \group{}$, and $\interval{t_i}{t_j} \in G_e$, with $t_i \leq t_n < t_j$. This means that the original plan added the group at time $t_i$ and removed it at $t_j$, with the possibility that $t_j = \forever$. In the latter case, there was no plan to remove the group originally.

\paragraph{Modularity}

As defined in Section~\vref{sec:scope}, the \textbf{removeGroup} operation's temporal scope is $\interval{t_n}{t_k}$, where $t_k$ is the time point in which the group was originally planned to be removed. We can see from the description above that $t_k = t_j$; the end point in the group's existence set containing $t_n$. We then have that the scope is defined as $\interval{t_n}{t_j}$. In the rule, we find the interval $\interval{t_i}{t_j}$ by looking up
\[
   \containing{G_e}{t_n} = \set{\interval{t_{e_1}}{t_{e_2}}}.
\]
According to the semantics of $\containing{\map{IS}}{t_n}$, it is true then that $\interval{t_i}{t_j} = \interval{t_{e_1}}{t_{e_2}}$, and so the temporal scope of the rule is $\interval{t_n}{t_{e_2}} = \interval{t_n}{t_j}$. Clearly, all time points looked up in the premise of the rule are contained within this interval, but the conclusion requires further examination. The $\names{}$ map is untouched and thus outside the scope. 

   The $\groups{}$ map is modified at key $\var{groupID}$ by assigning 
   $$\var{clampGroup}(\lookup{\groups}{\var{groupID}}, t_n)$$
   In $\var{clampGroup}$ (Figure~\vref{fun:clamp-group}), the intervals of the group's type and parent feature are clamped to end at $t_n$. These modifications stay within the temporal scope, as $\var{clampInterval}(\map{MAP}, t_c)$ clamps the mapping with an interval key containing $t_c$ to end at $t_c$. Due to \wf{8}, it is impossible that the group has a type or parent feature after $t_j$, which is the time point when the group was originally planned to be removed. Furthermore, the premise of the rule requires that $\lookup{G_t}{\interval{t_n}{t_j}} = \set{\var{type}}$ and $\lookup{G_p}{\interval{t_n}{t_j}} = \set{\var{parentFeatureID}}$, meaning that the group does not change its type or move during the temporal scope. Thus there is only one key in each of the group's type and parent feature maps containing $t_n$, and so the changed interval for these maps is $\interval{t_n}{t_j}$; the temporal scope. The existence interval is clamped in a similar way, and so stays within the temporal scope as well.

   Also, the $\features{}$ map is assigned $$\var{removeGroupAt}(\lookup{\features}{\var{parentFeatureID}}, \var{groupID}, t_n)$$ at key $\var{parentFeatureID}$. This helper function (Figure~\vref{fun:remove-group-at}) modifies the parent feature's set of child groups by calling 
   $$\var{clampIntervalValue}(F_c, t_c, \var{groupID})$$
   which behaves similarly to $\var{clampInterval}$ by clamping the interval containing $t_n$. The difference is that it removes only $\var{groupID}$ from the set of child groups, and adds the group to the set of child groups at the shortened interval. We conclude that this modification, too, happens within the temporal scope of the operation, as looking up any time point outside of the temporal scope will return the same results as the original plan.

   Recall that the spatial scope of the rule is the group itself, its parent feature, and its child features. The premise
\begin{equation*}
   \lookup{G_c}{\interval{t_n}{t_{e_2}}} = \emptyset
\end{equation*}
   ensures that the operation is not applied unless the group's set of child features is empty. The only features and groups looked up is the group itself and its parent feature. Thus, the rule stays within the spatial scope.

Based on the above proof, we conclude with the following lemma:
 \\

\begin{lemma}
   The $\rulefont{Remove-Group}$ rule operates strictly within the temporal and spatial scopes of the \textbf{removeGroup} operation.
   \label{lemma:remove-group-scope}
\end{lemma}

\paragraph{Preserving well-formedness}

Let
\begin{align*}
   \lookup{\groups}{\var{groupID}} &= \group \\
   \lookup{\groups'}{\var{groupID}} &= \groupvar{G'} \\
   \lookup{\features}{\var{parentFeatureID}} &= \feature \\
   \lookup{\features'}{\var{parentFeatureID}} &= \featurevar{F'}
\end{align*}

The \rulefont{Remove-Group} rule does not alter any feature's \textemdash{} in particular the root feature's \textemdash{} existence set or types map, and so \wf{1\textendash 2} hold for the modified model. It does however modify the child group map $F_c$, applying $\var{removeGroupAt}$ to the parent feature, the group's ID, and the removal time point. As previously argued, this function makes sure that the group ID is not in $\bigcup \lookup{F'_c}{\interval{t_n}{t_j}}$ \textemdash{} the parent feature's modified set of child groups \textemdash{} so \wf{3} holds.

Due to the semantics of $\var{clampGroup}$ and $\var{clampIntervalSet}$, no time points in the temporal scope are contained in an interval in the modified group's existence set ($\interval{t_n}{t_j} \notinnr G'_e$), so \wf{4}, \wf{6} and \wf{7} hold trivially. 

Since a premise of the rule is
   \[
      \lookup{G_c}{\interval{t_n}{t_{e_2}}} = \emptyset
   \]
   the group does not have any child features during the temporal scope in the original interval-based feature model. Due to the assumption that \wf{5} is true for the original model, no feature has $\var{groupID}$ listed as its parent group, so no feature is left without a parent group when the group is removed from the temporal scope. It follows that \wf{5} holds for the updated interval-based feature model as well. 

   As previously mentioned, $\interval{t_n}{t_j} \notinnr G'_e$, meaning that the group does not exist during the temporal scope in the modified model. For \wf{8} to hold for the updated model, we must then also have $\lookup{G'_t}{\interval{t_n}{t_j}} = \lookup{G'_p}{\interval{t_n}{t_j}} = \lookup{G'_c}{\interval{t_n}{t_j}} = \emptyset$. Recalling that $t_{e_2} = t_j$, then by definition of $\var{clampGroup}$ and the premise $\lookup{G_t}{\interval{t_n}{t_{e_2}}} = \set{\var{type}}$ in \rulefont{Add-Group}, we have that
\begin{align*}
   G'_t &= \var{clampInterval}\left(G_t, t_n\right) \\
        &= \var{clampInterval}\left(G''_t \cup \intervalmapping{t_{t_1}}{t_j}{\var{type}}, t_n\right) \\
        &= G''_t \cup \intervalmapping{t_{t_1}}{t_n}{\var{type}}
\end{align*}
Clearly, $\lookup{G''_t}{\interval{t_n}{t_j}} = \emptyset$.  Furthermore, since $\interval{t_{t_1}}{t_n}$ does not overlap $\interval{t_n}{t_j}$, $\lookup{G'_t}{\interval{t_n}{t_j}} = \emptyset$. An analogous argument can be made for $\lookup{G'_p}{\interval{t_n}{t_j}} = \emptyset$. From the definition of $\var{clampGroup}$, $G_c = G'_c$, so by the premise $\lookup{G_c}{\interval{t_n}{t_j}} = \emptyset$ in the rule, $\lookup{G_c'}{\interval{t_n}{t_j}} = \emptyset$. Consequently \wf{8} holds for the altered interval-based feature model.

Given that no cycles exist in the original model, removing a group does not create a new one, so \wf{9} holds.

We conclude that the \rulefont{Remove-Group} rule preserves well-formedness for the interval-based feature model, according to well-formedness rules \wf{1-9}. 
\\

\begin{lemma}
   The \rulefont{Remove-Group} rule preserves well-formedness of the interval-based feature model.
   \label{lemma:remove-group-well-formed}
\end{lemma}

\paragraph{Correctness of model modification}
% Vis: NAMES=NAMES' er uendret
% Vis: \var{groupID} \notin \bicup \lookup{F'_c}{\interval{t_n}{t_j}}
% Vis: \interval{t_n}{t_j} \notinnr G'_e
% \lookup{G_t}{\interval{t_n}{t_j}} = \lookup{G_p}{\interval{t_n}{t_j}} = \lookup{G_c}{\interval{t_n}{t_j}}
The semantics of the \textbf{removeGroup} operation dictate that the group should not exist, have a type, a parent, or child features after being removed. A proof for this can be found in the previous paragraph. Moreover, the parent feature's map of child features should not contain $\var{groupID}$ during the temporal scope. This is also proven in the previous paragraph. The $\names{}$ map should not be modified. It is clear from the rule that $\names{} = \names{}'$, so this condition, too, is true.

The above proof shows the following lemma:
\\

\begin{lemma}
   The \rulefont{Remove-Group} rule updates the interval-based feature model according to the semantics of the \textbf{removeGroup} operation.
   \label{lemma:remove-group-mod}
\end{lemma}
