\section{Soundness of the Remove Feature Rule}
\label{sec:soundness-of-the-remove-feature-rule}

See Figure~\vref{rule:remove-feature} for the \rulefont{Remove-Feature} rule. 
Let 
\begin{equation*}
   \begin{split}
      \textbf{removeFeature}(\var{featureID})\text{ at } t_n
      \shove \\
   (\names{}, \features{}, \groups{})
\end{split}
\end{equation*}
be the initial state, and
\[
   (\names'{}, \features'{}, \groups'{})
\] be the state after the \rulefont{Remove-Feature} rule has been applied. Recall that this operation removes the feature with ID \var{featureID} from the interval-based feature model $(\names{}, \features{}, \groups{})$ at $t_n$. Furthermore, let $\lookup{\features}{\var{featureID}} = \feature$, and $\interval{t_i}{t_j} \in F_e$, with $t_i \leq t_n < t_j$. This means that the original plan added the feature at time $t_i$ and removed it at $t_j$, with the possibility that $t_j = \forever$. In the latter case, there was no plan to remove the feature originally.

\paragraph{Modularity}
As defined in Section~\vref{sec:scope}, the \textbf{removeFeature} operation's temporal scope is $\interval{t_n}{t_k}$, where $t_k$ is the time point in which the feature was originally planned to be removed. We can see from the description above that $t_k = t_j$; the end point in the feature's existence set containing $t_n$. We then have that the scope is defined as $\interval{t_n}{t_j}$. In the rule, we find the interval $\interval{t_i}{t_j}$ by looking up
\begin{equation*}
   \containing{F_e}{t_n} = \set{\interval{t_{e_1}}{t_{e_2}}}
\end{equation*}
   According to the semantics of $\containing{\map{IS}}{t_n}$, it is true then that $\interval{t_i}{t_j} = \interval{t_{e_1}}{t_{e_2}}$, and so the temporal scope of the rule is $\interval{t_n}{t_{e_2}} = \interval{t_n}{t_j}$. Clearly, all time points looked up in the premise of the rule are contained within this interval, but the conclusion requires further examination. The $\names{}$ map is assigned $\var{clampInterval}(\lookup{\names}{\var{name}}, t_n)$ at key $\var{name}$. In $\var{clampInterval}$ (Figure~\vref{fun:clamp-interval}), the interval $\interval{t_{n_1}}{t_{n_2}}$ containing $t_n$ in $\lookup{\names}{\var{name}}$ is looked up and shortened to end at $t_n$ instead of $t_{n_2}$. This modification stays within the scope of the interval-based feature model, since the interval affected here is $\interval{t_n}{t_{n_2}}$, and necessarily, $t_{n_2} \leq t_j$, since the feature cannot possibly have a name after it is removed according to \wf{8}. 

   The $\features{}$ map is modified at key $\var{featureID}$ by assigning 
   $$\var{clampFeature}(\lookup{\features}{\var{featureID}}, t_n)$$
   In $\var{clampFeature}$ (Figure~\vref{fun:clamp-feature}), the intervals of the feature's name, type, and parent are clamped to end at $t_n$. These modifications, too, stay within the temporal scope, for the reason explained in the above paragraph. The existence interval is clamped in a similar way, and so stays within the temporal scope as well.

   Also, the $\groups{}$ map is assigned $$\var{removeFeatureAt}(\lookup{\groups}{\var{parentGroupID}}, \var{featureID}, t_n)$$ at key $\var{parentGroupID}$. This helper function (Figure~\vref{fun:remove-feature-at}) modifies the parent group's set of child features by calling 
   $$\var{clampIntervalValue}(G_c, t_c, \var{featureID})$$
   which behaves similarly to $\var{clampInterval}$ by clamping the interval containing $t_n$. The difference is that it removes only $\var{featureID}$ from the set of child features, and adds the feature to the set of child features at the shortened interval. We conclude that this modification, too, happens within the temporal scope of the operation, as looking up any time point outside of the temporal scope will return the same results as the original plan.

   Recall that the spatial scope of the rule is the feature itself, its parent group, and its child groups. The premise
\begin{equation*}
   \lookup{F_c}{\interval{t_n}{t_{e_2}}} = \emptyset
\end{equation*}
   ensures that the operation is not applied unless the feature's set of child groups is empty. The only features and groups looked up is the feature itself and its parent group. Thus, the rule stays within the spatial scope.

Based on the above proof, we conclude with the following lemma:
\\

\begin{lemma}
   The $\rulefont{Remove-Feature}$ rule operates strictly within the temporal and spatial scopes of the \textbf{removeFeature} operation.
   \label{lemma:remove-feature-scope}
\end{lemma}

\paragraph{Preserving well-formedness}
Let
\[
   \lookup{\features'}{\var{featureID}} = \featurevar{F'}
\]


The \rulefont{Remove-Feature} rule contains the premise
\begin{equation*}
   \lookup{F_p}{\interval{t_n}{t_{e_2}}} = \set{\var{parentGroupID}}
\end{equation*}
ensuring that the feature has \emph{exactly} one parent group during the temporal scope of the rule. Under the assumption that \wf{1} holds in the original model, the feature being removed cannot be the root feature, since the root has no parent group. Furthermore, it means that the feature does not move during the temporal scope, which would be a conflict. Therefore, both \wf{1} and \wf{2} hold in the modified model.

For any time point $t_n$ in the temporal scope of the rule, $t_n \notinn F'_e$ due to the semantics of $\var{clampFeature}$ (Figure~\vref{fun:clamp-feature}), so \wf{3} holds trivially for the updated model.  The only change made to a group is by the function $\var{removeFeatureAt}$ (Figure~\vref{fun:remove-feature-at}), which removes the feature from the parent group's map of child groups during $\interval{t_n}{t_j}$. Hence \wf{5} holds, and since that function does not modify the types map $G_t$ of the group, \wf{4} holds given that it is true for the original model. 

The premise $\lookup{F_c}{\interval{t_n}{t_{e_2}}} = \emptyset$ ensures that the feature to be removed does not have any child groups during the temporal scope, so no group is left without a parent in the updated model. Thus \wf{6} holds. 

Suppose that the parent group has the type $\xortype{}$ or $\ortype{}$ at some point during the temporal scope. In the original model, no child feature of the group has type $\mandatory{}$ due to the assumption that \wf{7} is true. The \rulefont{Remove-Feature} rule does not add any features or change a feature type, so this requirement still holds for the modified model. 

After applying the rule, we have that $\interval{t_n}{t_j} \notinnr F'_e$, which means that the feature does not exist during the temporal scope of the operation. To fulfil \wf{8}, we must furthermore have that $\lookup{F_n}{\interval{t_n}{t_j}} = \lookup{F_t}{\interval{t_n}{t_j}}= \lookup{F_p}{\interval{t_n}{t_j}} = \lookup{F_c}{\interval{t_n}{t_j}} = \emptyset$, and that $\var{featureID} \notin \lookup{\lookup{\names}{\var{name}}}{\interval{t_n}{t_j}}$. The former statement holds due to the semantics of $\var{clampFeature}$ and $\var{clampInterval}$; the feature's attributes are all clamped to end at the time of removal, and the premises on the form $\lookup{F_x}{\interval{t_n}{t_{e_2}}} = \set{v}$ ensure that no changes are made to those attributes during the temporal scope. $\lookup{F_c}{\interval{t_n}{t_j}} = \emptyset$ is a premise in the rule (since $t_j = t_{e_2}$). As for the $\names$ map, the mapping from $\var{name}$ to $\intervalmapping{t_i}{t_j}{\var{featureID}}$ is replaced by $\intervalmapping{t_i}{t_n}{\var{featureID}}$ in the function $\var{clampInterval}(\lookup{\names}{\var{name}}, t_n)$. Hence \wf{8} is true for the altered interval-based feature model.

Under the assumption that no cycles exist in the original model, removing a feature does not create a new one, so \wf{9} is holds for the modified model as well.

We conclude that the \rulefont{Remove-Feature} rule preserves well-formedness for the interval-based feature model, according to well-formedness rules \wf{1-9}. 
\\

\begin{lemma}
   The \rulefont{Remove-Feature} rule preserves well-formedness of the interval-based feature model. 
   \label{lemma:remove-feature-well-formed}
\end{lemma}

\paragraph{Correctness of model modification}
The semantics of the \textbf{removeFeature} operation is that applying it should remove the feature from the plan from $t_n$ until the point at which is was originally planned to be removed. Then if $\lookup{\features'}{\var{featureID}} = \featurevar{}$, and $\containing{F_e}{t_n} = \interval{t_i}{t_j}$, then
\begin{align}
   \interval{t_n}{t_j} &\notinnr F_e && \text{the feature does not exist} \tag{1} \label{remove-feature1}\\
   \lookup{F_n}{\interval{t_n}{t_j}} &= \emptyset && \text{the feature has no name} \tag{2} \label{remove-feature2}\\
   \lookup{F_t}{\interval{t_n}{t_j}} &= \emptyset  && \text{the feature has no type} \tag{3} \label{remove-feature3}\\
   \lookup{F_p}{\interval{t_n}{t_j}} &= \emptyset  && \text{the feature has no parent group} \tag{4} \label{remove-feature4}\\
   \lookup{F_c}{\interval{t_n}{t_j}} &= \emptyset && \text{the feature has no child groups} \tag{5} \label{remove-feature5}
\end{align}
Since we established $\interval{t_n}{t_j} \notinnr F_e$ in the above paragraph, these
statements follow directly from Lemma~\vref{lemma:remove-feature-well-formed} and \wf{8}. It further follows that no name is associated with $\var{featureID}$ in the $\names{}'$ map, and that no group in the $\groups{}'$ map has the feature listed as a child feature.

The above proof shows the following lemma:
\\

\begin{lemma}
   The \rulefont{Remove-Feature} rule updates the interval-based feature model according to the semantics of the \textbf{removeFeature} operation.
   \label{lemma:remove-feature-mod}
\end{lemma}
