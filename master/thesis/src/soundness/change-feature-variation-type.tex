
\subsection{Soundness of the Change feature variation type rule} 
\label{sub:soundness-of-the-change-feature-variation-type-rule}

See figure \vref{rule:change-feature-varation-type} for the \rulefont{Change-Feature-Variation-Type} rule. Let 
\begin{equation*}
   \begin{split}
      \textbf{changeFeatureVariationType}(\var{featureID}, \var{type}) \text{ at } t_n \shove \\
      (\names, \features, \groups)
   \end{split}
\end{equation*}
be the initial state, and
\[
   (\names', \features', \groups')
\]
be the result state after applying the \rulefont{Change-Feature-Variation-Type} rule. Recall that this operation changes the type of the feature with ID $\var{featureID}$ to $\var{type}$.

\paragraph{Scope}
Recall that the temporal scope of the move-group rule is $\interval{t_n}{t_k}$ (section \vref{sec:define-the-scope}), where $t_k$ is the time point at which the type is originally planned to be change or the feature is removed. In the rule, this scope is identified by 
   \[
      \containing{F_t}{t_n} = \set{\interval{t_{t_1}}{t_{t_2}}}
   \]
   Here, the time point $t_n$ for changing the feature type is looked up in the feature's types map's set of interval keys, and the expected result is $\set{\interval{t_{t_1}}{t_{t_2}}}$. This means that there is a mapping $\intervalmapping{t_{t_1}}{t_{t_2}}{\var{oldType}}$ in $F_t$, with $\var{oldType}$ being the type of the feature at time $t_n$, and this stops being the case at $t_{t_2}$. Thus the temporal scope of this operation is $\interval{t_n}{t_{t_2}}$. The only interval looked up or assigned to in the rule is $\interval{t_n}{t_{t_2}}$, so the rule operates strictly within the temporal scope of the operation.

   The spatial scope for this operation is the feature itself and its parent group. Since the feature may move during the temporal scope, there may be several parent groups to consider. These groups and their types are looked up in the premise
\[
   \begin{split}
      \forall \interval{t_{p_1}}{t_{p_2}} \in \overlapping{F_p}{t_n}{t_{t_2}}  \\ % for all parent intervals overlapping the temporal scope
      \forall p \in \lookup{F_p}{\interval{t_{p_1}}{t_{p_2}}}  \\% for all parent groups (always exactly one) in the parent interval
      \forall t \in \, \var{getTypes}\left(\lookup{\groups}{p}, \clamp{\interval{t_{p_1}}{t_{p_2}}}{t_n}{t_{t_2}}\right)  \\
          \big(\var{compatibleTypes}(t, \var{type})\big) 
    \end{split}
\]

Otherwise, the only feature or group looked up or assigned to in the rule is $\lookup{features}{\var{featureID}}$, so the rule stays within the spatial scope.

\begin{lemma}
   The \rulefont{Change-Feature-Variation-Type} rule operates strictly within the scope of the \textbf{changeFeatureVariationType} operation.
   \label{lemma:change-feature-variation-type-scope}
\end{lemma}

\paragraph{Preserving well-formedness}
\todo{this}
\\

\begin{lemma}
   The \rulefont{Change-Feature-Variation-Type} rule preserves well-formedness of the temporal feature model. 
   \label{lemma:change-feature-variation-type-well-formed}
\end{lemma}

\paragraph{Correctness of model modification}
\todo{this too}
\\

\begin{lemma}
   The \rulefont{Change-Feature-Variation-Type} rule updates the temporal feature model according to the semantics of the \textbf{changeFeatureVariationType} operation.
   \label{lemma:change-feature-variation-type-mod}
\end{lemma}
