\subsection{Soundness of the Change Feature Name Rule} 
\label{sub:soundness-of-the-change-feature-name-rule}

See Figure~\vref{rule:change-feature-name} for the \rulefont{Change-Feature-Variation-Type} rule. Let 
\begin{equation*}
   \begin{split}
      \textbf{changeFeatureName}(\var{featureID}, \var{name}) \text{ at } t_n \shove \\
      (\names, \features, \groups)
   \end{split}
\end{equation*}
be the initial state, and
\[
   (\names', \features', \groups')
\]
be the result state after applying the \rulefont{Change-Feature-Name} rule. Recall that this operation changes the name of the feature with ID $\var{featureID}$ to $\var{name}$.

\paragraph{Scope}
Recall that the temporal scope of the \rulefont{Change-Feature-Name} rule is $\interval{t_n}{t_k}$ (Section~\vref{sec:scope}), where $t_k$ is the time point at which the name is originally planned to be changed or the feature is removed. In the rule, this scope is identified by 
   \[
      \containing{F_n}{t_n} = \set{\interval{t_{n_1}}{t_{n_2}}}
   \]
   Here, the time point $t_n$ for changing the name is looked up in the feature's names map's set of interval keys, and the expected result is $\set{\interval{t_{n_1}}{t_{n_2}}}$. This means that there is a mapping $\intervalmapping{t_{n_1}}{t_{n_2}}{\var{oldName}}$ in $F_n$, with $\var{oldName}$ being the name of the feature at time $t_n$, and this stops being the case at $t_{n_2}$. Thus the temporal scope of this operation is $\interval{t_n}{t_{n_2}}$. The only interval looked up or assigned to in the rule is $\interval{t_n}{t_{n_2}}$, so the rule operates strictly within the temporal scope of the operation.

   The spatial scope for this operation is the name, the feature, and its original name. The only feature looked up or assigned to is $\lookup{\features}{\var{featureID}}$, and the only names looked up or assigned to are $\var{oldName}$ and $\var{name}$. The $\groups{}$ map is not modified or looked up in by the rule. Clearly, the rule stays within the spatial scope.
\\

\begin{lemma}
   The \rulefont{Change-Feature-Name} rule operates strictly within the scope of the \textbf{changeFeatureName} operation.
   \label{lemma:change-feature-name-scope}
\end{lemma}

\paragraph{Preserving well-formedness}
The rule does not modify any feature's existence set or type, so \wf{1\textendash 2} holds. Since it does change a name, we must look at that modification to make sure that \wf{3} is true for the altered model. A requirement for \wf{3} is that a feature has \emph{exactly} one name. 
The feature is altered thus:
\[
   \begin{split}
        (\lookup{\left(\lookup{\lookup{\names{}}{\var{name}}}{\interval{t_n}{t_{n_2}}} \assign \var{featureID}\right)}{\var{oldName}} \assign \\
        \var{clampInterval}(\lookup{\names}{\var{oldName}}, t_n), \\
   \end{split}
\]
This ensures that the feature's original name stops at $t_n$ and the new one lasts for the duration of the temporal scope $\interval{t_n}{t_{n_2}}$, ensuring that the feature has \emph{exactly} one name during the temporal scope.
Moreover \wf{3} requires that the name belongs to the same feature, and no other. This is fulfilled by 
\[
   \begin{split}
        \lookup{\lookup{\big(\lookup{\names{}}{\var{oldName}} \assign 
        \var{clampInterval}\left(\lookup{\names}{\var{oldName}}, t_n\right) \big)}{\var{name}}}{\interval{t_n}{t_{n_2}}} \\
        \assign \var{featureID} \\
   \end{split}
\]
Here, the interval containing $t_n$ in $\lookup{\names}{\var{oldName}}$ is clamped to end at $t_n$, and the resulting map is assigned $\var{featureID}$ at $\var{name}$ during the temporal scope, so the new name belongs to only the feature. This fulfils \wf{3}.
As no other part of the interval-based feature model is modified, $\wf{4\textendash 9}$ hold.
\\

\begin{lemma}
   The \rulefont{Change-Feature-Name} rule preserves well-formedness of the interval-based feature model. 
   \label{lemma:change-feature-name-well-formed}
\end{lemma}

\paragraph{Correctness of model modification}
The expected result of applying the rule is that $\lookup{\features'}{\var{featureID}} = \featurevar{F'}$ has the name $\var{name}$ during the temporal scope $\interval{t_n}{t_{n_2}}$. Indeed, due to the semantics of $\var{clampInterval}$ and assignment, for any time point $t_i$ such that $t_n \leq t_i < t_{n_2}$,
\[
   \lookup{F'_n}{t_i} = \set{\var{name}}
\]
Additionally, we should have $\lookup{\names'}{\var{name}} = \var{featureID}$. This is shown in the previous paragraph on well-formedness.
Since no other part of the interval-based feature model is altered, the rule performs as desired.
\\

\begin{lemma}
   The \rulefont{Change-Feature-Name} rule updates the interval-based feature model according to the semantics of the \textbf{changeFeatureName} operation.
   \label{lemma:change-feature-name-mod}
\end{lemma}
