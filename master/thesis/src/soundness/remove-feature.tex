\section{Soundness of the Remove feature rule}
\label{sec:soundness-of-the-remove-feature-rule}

See figure \vref{rule:remove-feature} for the \rulefont{Remove-Feature} rule. 
Let 
\begin{equation*}
   \begin{split}
      \textbf{removeFeature}(\var{featureID})\text{ at } t_n
      \shove \\
   (\names{}, \features{}, \groups{})
\end{split}
\end{equation*}
be the initial state. Recall that this operation removes the feature with ID \var{featureID} from the temporal feature model $(\names{}, \features{}, \groups{})$ at $t_n$. Furthermore, let $\lookup{\features}{\var{featureID}} = \feature$, and $\interval{t_i}{t_j} \in F_e$, with $t_i \leq t_n < t_j$. This means that the original plan added the feature at time $t_i$ and removed it at $t_j$, with the possibility that $t_j = \forever$. In the latter case, there was no plan to remove the feature originally.

\paragraph{Scope}
As defined in section \vref{sec:define-the-scope}, the \textbf{removeFeature} rule's temporal scope is $\interval{t_n}{t_k}$, where $t_k$ is the time point in which the feature was originally planned to be removed. We can see from the description above that $t_k = t_j$; the end point in the feature's existence set containing $t_n$. We then have that the scope is defined as $\interval{t_n}{t_j}$. In the rule, we find the interval $\interval{t_i}{t_j}$ by looking up
\begin{equation*}
   \containing{F_e}{t_n} = \set{\interval{t_{e_1}}{t_{e_2}}}
\end{equation*}
   According to the semantics of $\containing{\map{IS}}{t_n}$, it is true then that $\interval{t_i}{t_j} = \interval{t_{e_1}}{t_{e_2}}$, and so the temporal scope of the rule is $\interval{t_n}{t_{e_2}} = \interval{t_n}{t_j}$. Clearly, all time points looked up in the premise of the rule are contained within this interval, but the conclusion requires further examination. The $\names{}$ map is assigned $\var{clampInterval}(\lookup{\names}{\var{name}}, t_n)$ at key $\var{name}$. In $\var{clampInterval}$ (figure \vref{fun:clamp-interval}), the interval $\interval{t_{n_1}}{t_{n_2}}$ containing $t_n$ in $\lookup{\names}{\var{name}}$ is looked up and shortened to end at $t_n$ instead of $t_{n_2}$. This modification stays within the scope of the temporal feature model, since the interval affected here is $\interval{t_n}{t_{n_2}}$, and necessarily, $t_{n_2} \leq t_j$, since the feature cannot possibly have a name after it is removed according to \itbf{TFMWF8}. 

   The $\features{}$ map is modified at key $\var{featureID}$ by assigning $\var{clampFeature}(\lookup{\features}{\var{featureID}}, t_n)$. In $\var{clampFeature}$ (figure \vref{fun:clamp-feature}), the intervals of the feature's name, type, and parent are clamped to end at $t_n$. These modifications, too, stay within the temporal scope, for the reason explained in the above paragraph. The existence interval is clamped in a similar way, and so stays within the temporal scope as well.

   Also, the $\groups{}$ map is assigned $\var{removeFeatureAt}(\lookup{\groups}{\var{parentGroupID}}, \var{featureID}, t_n)$ at key $\var{parentGroupID}$. This helper function (figure \vref{fun:remove-feature-at}) modifies the parent group's table of subfeatures by calling $\var{clampIntervalValue}(G_c, t_c, \var{featureID})$, which behaves similarly to $\var{clampInterval}$ by clamping the interval containing $t_n$. The difference is that it removes only $\var{featureID}$ from the set of child groups, and adds the feature to the set of subfeatures at the shortened interval. We conclude that this modification, too, happens within the temporal scope of the operation, as looking up any time point outside of the temporal scope will return the same results as the original plan.

   Recall that the spatial scope of the rule is the feature itself, its parent group, and its subtree. The premise
\begin{equation*}
   \lookup{F_c}{\interval{t_n}{t_{e_2}}} = \emptyset
\end{equation*}
   ensures that the operation is not applied unless the feature's subtree is empty. The only features and groups looked up is the feature itself and its parent group. Thus, the rule stays within the spatial scope.

\begin{lemma}
   The $\rulefont{Remove-Feature}$ rule operates strictly within the scope of the \textbf{removeFeature} operation.
\end{lemma}

\paragraph{Preserving well-formedness}

The \rulefont{Remove-Feature} rule contains the premise
\begin{equation*}
   \lookup{F_p}{\interval{t_n}{t_{e_2}}} = \set{\var{parentGroupID}}
\end{equation*}
, ensuring that the feature has \emph{exactly} one parent group during the temporal scope of the rule. Under the assumption that \itbf{TFMWF1} holds in the original model, the feature being removed cannot be the root feature, since the root has no parent group. Furthermore, it means that the feature does not move during the temporal scope, which would be a conflict. Therefore, both \itbf{TFMWF1} and \itbf{TFMWF2} hold in the modified model.

For any time point $t_n$ in the temporal scope of the rule, $t_n \notinn F_e$, so \itbf{TFMWF3} holds trivially. Since $\var{removeFeatureAt}(\lookup{\groups}{\var{parentGroupID}}, \var{featureID}, t_n)$ is the only change made to a group, and since it only 

