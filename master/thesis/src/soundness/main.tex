\chapter{Soundness}
\label{cha:soundness}
% \todo{A biiiggg issue with these proofs is that I don't differentiate between the feature model before $(\names, \features, \groups)$ and modified $(\names', \features', \groups')$.}

% \todo{and completeness? The operations in themselves are not complete, in the sense that there is not a one-to-one correspondence between the operations defined on interval-based feature models (temporal operations) and the operations defined on feature models (model operations). However, the temporal operations are more restricted than the model operations, and any valid change within that subset to an evolution plan would be accepted by the rules. Should I try to prove completeness as well?}

\todo{Move to appendix, but keep a few (maybe 2) examples to show how they work}
\todo{Prepare the reader for what is to come. To not overwhelm the reader with proofs, the details can be found in the appendix. Here I will emphasize how I prove things, give concrete samples, and point to where the rest of it can be found.}

We use an inductive proof structure to prove soundness and modularity for the rule system (Section~\vref{sec:sos-rules}). The base case is a proof by construction, in which we define a sound interval-based feature model. Using the induction hypothesis that the initial model is sound, we prove that each rule preserves soundness and operates within the previously defined scope (Section~\vref{sec:define-the-scope}).

\section{Soundness for interval-based feature models}
\label{sec:soundness-for-interval-based-feature-models}
% \todo{Argue that the rules are \emph{enough} to preserve soundness.}
% \todo{Show what needs to be proved for an initial model. ``correct by construction". In a well-formed structure. Lemma: What is a sound structure. Soundness theorem: Preserves soundness, using the lemma.}
% \todo{Proof for each rule should be lemma, and the whole system is a theorem or something:). Remember to number the lemmas}
% \todo{The structure of the proof should be introduced first}
% \todo{In the theorem, emphasize the structure of the proof (inductive)}
% \todo{Between the proofs, add some text; how to read the proof, how to understand etc. Also give context (how the lemmas are used later)}
The interval-based feature model can be viewed as a sequence of feature models associated with time points. A feature model has strict structural requirements, and the definition of a paradox is a feature model that violates these requirements. In this context, soundness means that if a rule accepts a modification, realising the modified plan results in a sequence of feature models where each is structurally sound. The soundness analysis in this chapter assumes that the original plan is sound; i.e. all resulting feature models fulfil the structural requirements. 

We must first define what it means for an interval-based feature model to be sound. Essentially, it means that if we converted the interval-based feature model into a sequence of time points associated with feature models, each feature model would be sound.


The well-formedness requirements listed in Section \vref{sub:feature-models} can be translated into rules for interval-based feature models $(\names, \features, \groups)$. We assume that the first time point in the plan is $t_0$.

\begin{enumerate}[\wf{\arabic*}, itemsep=0mm]
   \item An interval-based feature model has exactly one root feature. We assume that the root ID is $\var{rootID}$, and that $\lookup{\features}{\var{rootID}} = \featurevar{R}$. This also means that $R_e=\set{\interval{t_0}{\forever}}$ \textemdash{} the root always exists, and that $R_p = \emptyset$ \textemdash{} the root never has a parent group.
\item The root feature must be mandatory. This means that $$R_t  = \set{\intervalmapping{t_0}{\forever}{\mandatory}}$$ where $R_t$ is the types map of the root feature. 
\item At any time $t_n \geq t_0$, each feature has exactly one unique name, variation type and (potentially empty) collection of subgroups. Given a feature ID $\var{featureID}$, this means that if $\lookup{\features}{\var{featureID}} = \feature$ and $t_n \inn F_e$ , then
   \begin{enumerate}[(i)]
      \item $\lookup{F_n}{t_n} = \set{\var{name}}$ \textemdash{} the feature has exactly one name,
      \item $\lookup{\lookup{\names}{\var{name}}}{t_n} = \set{\var{featureID}}$ \textemdash{} the name is unique \emph{at the time point $t_n$},
      \item $\lookup{F_t}{t_n} = \set{\var{type}} \text{ with } \var{type} \in \set{\mandatory, \optional}$ \textemdash{} the feature has exactly one type, and
      \item $\lookup{F_c}{t_n} = C$, such that $\bigcup C$ is a set of the group IDs, and if $\var{groupID} \in \bigcup C$ and $\lookup{\groups}{\var{groupID}} = \group{}$, then $\lookup{G_p}{t_n} = \set{\var{featureID}}$ \textemdash{} if a group is listed as a subgroup of a feature, then the feature is listed as the parent of the group at the same time.
   \end{enumerate}
   \item At any time $t_n \geq t_0$, each group has exactly one variation type. Given a group ID $\var{groupID}$, this means that if $\lookup{\groups}{\var{groupID}} = \group$ and $t_n \inn G_e$, then $\lookup{G_t}{t_n} = \set{\var{type}}$ for $\var{type} \in \set{\andtype, \ortype, \xortype}$.
   \item At any time $t_n \geq t_0$, each feature, except for the root feature, must be part of exactly one group. Formally, given a feature ID $\var{featureID} \neq \var{rootID}$, if $\lookup{\features}{\var{featureID}} = \feature$, and $t_n \inn F_e$, then $\lookup{F_p}{t_n} = \set{\var{groupID}}$ with $\lookup{\groups}{\var{groupID}} = \group$, $t_n \inn G_e$, and $\var{featureID} \in \bigcup \lookup{G_c}{t_n}$. Conversely, if $\var{featureID} \in \bigcup \lookup{G_c}{t_n}$, then $\lookup{F_p}{t_n} = \var{groupID}$.
   \item At any time $t_n \geq t_0$, each group must have exactly one parent feature. Formally, given a group ID $\var{groupID}$, if $\lookup{\groups}{\var{groupID}} = \group$ and $t_n \inn G_e$, then $\lookup{G_p}{t_n} = \set{\var{featureID}}$, and $\lookup{\features}{\var{featureID}} = \feature$ with $\var{groupID} \in \bigcup \lookup{F_c}{t_n}$.
   \item At any time $t_n$, a group with types \xortype{} or \ortype{} must not contain \mandatory{} features. Formally, given a group ID $\var{groupID}$ with $\lookup{\groups}{\var{groupID}} = \group$, if $\lookup{F_t}{t_n} = \set{\var{type}}$ with $\var{type} \in \set{\xortype{}, \ortype{}}$, and if $\var{featureID} \in \bigcup \lookup{F_c}{t_n}$ and $\lookup{\features}{\var{featureID}} = \feature$, then $\lookup{F_t}{t_n} = \set{\optional}$.
\end{enumerate}

We must add two additional requirements:
\begin{enumerate}[itemsep=0mm]
   \item[\wf{8}] For a feature with ID $\var{featureID}$ such that $\lookup{\features}{\var{featureID}} = \feature$, if $t_n \notinn F_e$, then $\lookup{F_n}{t_n} = \lookup{F_t}{t_n} = \lookup{F_p}{t_n} = \lookup{F_c}{t_n} = \emptyset$, and for all keys $\var{name}$ in $\names$, $\var{featureID} \notin \lookup{\lookup{\names}{\var{name}}}{t_n}$ \textemdash{} no name belongs to the feature. Similarly, for a group with ID $\var{groupID}$ such that $\lookup{\groups}{\var{groupID}} = \group$, if $t_n \notinn G_e$, then $\lookup{G_t}{t_n} = \lookup{G_p}{t_n} = \lookup{G_c}{t_n} = \emptyset$. In other words, a feature or a group which does not exist cannot have a name, a type, a parent, or a child.
   \item[\wf{9}] The interval-based feature model contains no cycles, which means that at any time point $t_n \geq t_0$, for any feature or group that exists at $t_n$, if we follow the parent chain upwards, we never encounter the same feature or group twice. In other words, no feature or group is its own ancestor.
\end{enumerate}

Together, these requirements form the basis of the soundness proofs. We assume that the original plan is sound, so each of these requirements is assumed to be true for the original interval-based feature model. Furthermore, we prove that the requirements must still hold for the updated model if the rule can be applied.

\section{Soundness of each rule}
\label{sec:soundness-of-each-rule}
In the following sections, we prove that each rule is sound, and conclude that the system is sound.

For each rule, the proof for soundness includes three parts: 
\begin{enumerate}[(i)]
   \item proving that the rule operates strictly within the previously defined temporal and spatial scopes (Section~\vref{sec:define-the-scope}), 
   \item that the rule preserves well-formedness, as defined in the above requirements \wf{1\textendash 9}, and
   \item that the rule updates the model correctly, preserving soundness as well as respecting the semantics of the operation.
\end{enumerate}



\section{Soundness of the Add feature rule}
\label{sec:soundness-of-the-add-feature-rule}

\todo{Use a proof environment that clearly separate definitions from proofs. The proofs should contain syntactic correctness, not semantic (that part is moved out of the proof)}

See figure \vref{rule:add-feature} for the \rulefont{Add-Feature} rule.
Let 
\begin{equation*}
   \begin{split}
      \textbf{addFeature}(\var{featureID}, \var{name}, \var{type}, \var{parentGroupID})\text{ at }\interval{t_n}{t_m}
      \shove \\
   (\names{}, \features{}, \groups{})
\end{split}
\end{equation*}
be the initial state. Recall that this operation adds the feature with ID \var{featureID} to the temporal feature model $(\names{}, \features{}, \groups{})$ from $t_n$ to $t_m$. We assume that $(\names{}, \features{}, \groups{})$ is well-formed, as defined in \itbf{TFMWF1-9}.

\todo{Show that the feature model outside the scope is unchanged}

\paragraph{Scope}

Recall from section \vref{sec:define-the-scope} that the temporal scope of this operation is $\interval{t_n}{t_m}$, and the spatial scope is the feature itself, the parent group and the name. 

In the rule, we look up only the feature ID, the parent group ID, and the name, and update only the name, feature, and parent group. Thus, the rule operates within the spatial scope of the operation. Furthermore, the only interval looked up in the interval maps and sets of the model is $\interval{t_n}{t_m}$, which is exactly the temporal scope of the rule. Hence the rule operates strictly within the temporal and spatial scopes of the operation.
\\

\begin{lemma}
   The \rulefont{Add-Feature} rule operates strictly within the scope of the \textbf{addFeature} operation.
   \label{lemma:add-feature-scope}
\end{lemma}


\paragraph{Preserving well-formedness}

If the rule is applied, the well-formedness requirements must hold for the updated feature model. 

Since the rule checks that the feature does not already exist during the temporal scope, it is impossible that $\var{featureID} = \var{rootID}$. Thus the rule does not affect the root feature, and \itbf{TFMWF1} and \itbf{TFMWF2} hold for the updated temporal feature model.

Because we assume that \itbf{TFMWF8} holds for the original model, and the feature does not exist during $\interval{t_n}{t_m}$, the feature has no name, type, or subgroups in the original plan. When we add the feature to the feature model using \var{setFeatureAttributes}, we give the feature exactly one name and one type during the temporal scope, and the set of child groups is empty. The temporal scope is also added to the feature's existence set, so only the new feature has the ID $\var{featureID}$ during the temporal scope. To link the feature ID to the name, the rule sets the feature ID as the value at key $\var{name}$ in the $\names{}$ map during the temporal scope, so the name is unique during the temporal scope.
Consequently, \itbf{TFMWF3} holds.

The rule does not modify the parent group's variation type, so \itbf{TFMWF4} is preserved in the modified temporal feature model.

Similarly to the argument for \itbf{TFMWF3}, the parent group ID is uniquely defined for the feature in \var{setFeatureAttributes}, and \var{featureID} is added to the parent group's set of child features, so the new feature is part of exactly one group. Since we do not remove any other feature IDs from the parent group's set of features, and as we already established that the new feature is not the root feature, \itbf{TFMWF5} is preserved.

The new feature does not have any subgroups during the temporal scope, and we do not modify the parent group's parent feature. Under the assumption that \itbf{TFMWF6} holds in the original model, it still holds after applying the \rulefont{Add-Feature} rule.

The rule verifies that all of the parent group's types are compatible with the added feature's type during the temporal scope, so \itbf{TFMWF7} holds after applying the rule.

Since the rule adds the temporal scope to the new feature's existence table, and since the parent group exists in the original plan, \itbf{TFMWF8} is preserved after the rule is applied.

It is furthermore impossible that adding this feature creates a cycle in the modified model. The new feature has no subgroups, so it cannot be part of a cycle. Because of the assumption that \itbf{TFMWF9} holds in the original plan, and applying the rule does not introduce a cycle, this requirement still holds. 

As the rule operates within the scope (lemma \vref{lemma:add-feature-scope}), it does not affect any other part of the plan.

We conclude that the \rulefont{Add-Feature} rule preserves well-formedness for the temporal feature model, according to well-formedness rules \itbf{TFMWF1-9}. 
\\
\begin{lemma}
   The \rulefont{{Add-Feature}} rule preserves well-formedness of the temporal feature model.
   \label{lemma:add-feature-well-formed}
\end{lemma}

\todo{Mention that the scope part ensures that no other part of the plan is affected}
\paragraph{Correctness of model modification}

The operation is intended to add the feature with ID \var{featureID} to the temporal feature model during the interval $\interval{t_n}{t_m}$. 

After adding the feature to the temporal feature model, looking up the name $\var{name}$ in the $\names{}$ map at interval key $\interval{t_n}{t_m}$ should give the value $\var{featureID}$. Indeed, since the $\names{}$ map is updated thus:
\[
   \lookup{\lookup{\names}{\var{name}}}{\interval{t_n}{t_m}} \assign \var{featureID}
\]
, then due to the semantics of map assignment (definition \vref{def:map}), and lookup in interval maps (definition \vref{def:interval-map}),
\[
   \lookup{\lookup{\names}{\var{name}}}{\interval{t_n}{t_m}} = \set{\var{featureID}}
\]
will hold.

Similarly, if we wish to lookup information about the feature during the interval $\interval{t_n}{t_m}$ in the modified model, the results should match the information in the operation. The rule assigns 
\begin{equation*}
   \begin{split}
   \var{setFeatureAttributes}(\lookup{\features{}}{\var{featureID}}, \interval{t_n}{t_m}, \var{name}, \var{type}, \\
   \var{parentGroupID})
   \end{split}
\end{equation*}

to $\lookup{\features}{\var{featureID}}$. 

According to the semantics of assignment (section \vref{sub:maps}) and $\var{setFeatureAttributes}$ (figure \vref{fun:set-feature-attributes}), and given that $\lookup{\features}{\var{featureID}} = \feature$, then 
\begin{align}
   \interval{t_n}{t_m} &\in F_e && \text{the feature exists} \tag{1} \label{add-feature1}\\
   \lookup{F_n}{\interval{t_n}{t_m}} &= \set{\var{name}} && \text{the feature has the expected name} \tag{2} \label{add-feature2}\\
   \lookup{F_t}{\interval{t_n}{t_m}} &= \set{\var{type}}  && \text{the feature has the expected type} \tag{3} \label{add-feature3}\\
   \lookup{F_p}{\interval{t_n}{t_m}} &= \set{\var{parentGroupID}}  && \text{the feature has the expected parent group} \tag{4} \label{add-feature4}\\
   \lookup{F_c}{\interval{t_n}{t_m}} &= \emptyset && \text{the feature has no children} \tag{5} \label{add-feature5}
\end{align}

Statement (\ref{add-feature1}) holds due to the line $F_e \cup \interval{t_n}{t_m}$ in $\var{setFeatureAttributes}$. The next four hold due to both premises in the rule and modifications in the function. Due to the premise $\interval{t_n}{t_m} \not \innr F_e$, which means that the feature does not previously exist at any point during the interval, and since \itbf{TFMWF8} is assumed to hold for the original model, the original feature does not have a name, type, parent group or child groups during the interval. In the function $\var{setFeatureAttributes}$, the name is added ($\lookup{F_n}{\interval{t_{start}}{t_{end}}} \assign \var{name}$), and so is the type ($\lookup{F_t}{\interval{t_{start}}{t_{end}}} \assign \var{type}$) and the parent group ($\lookup{F_t}{\interval{t_{start}}{t_{end}}} \assign \var{parentGroupID}$). The child groups are not modified, and so (\ref{add-feature5}) holds.

The child features of the group must also be updated according to the semantics of the operation. After applying the rule, given that $\lookup{\groups}{\var{parentGroupID}} = \group{}$, 
\[
   \var{featureID} \in \bigcup \lookup{G_c}{\interval{t_n}{t_m}}
\]
, meaning that the feature is in the parent group's set of child features in the updated model. This holds because $\lookup{\groups}{\var{parentGroupID}}$ is assigned $\var{addChildFeature}(\lookup{\groups}{\var{parentGroupID}}, \interval{t_n}{t_m}, \var{featureID})$ (see figure \vref{fun:add-child-feature}), which modifies $G_c$ by adding $\var{featureID}$ to the set of child features at interval key $\interval{t_n}{t_m}$.
\\

\begin{lemma}
   The \rulefont{Add-Feature} rule updates the temporal feature model according to the semantics of the \textbf{addFeature} operation.
   \label{lemma:add-feature-mod}
\end{lemma}




\subsection{Soundness for the Add group rule}
\label{sub:soundness-for-the-add-group-rule}

See figure \vref{rule:add-group} for the \rulefont{Add-Group} rule. 
Let 
\begin{equation*}
   \begin{split}
      \textbf{addGroup}(\var{groupID}, \var{type}, \var{parentFeatureID})\text{ at }\interval{t_n}{t_m}
      \shove \\
   (\names{}, \features{}, \groups{})
\end{split}
\end{equation*}
be the initial state. Recall that this operation adds the group with ID \var{groupID} to the temporal feature model $(\names{}, \features{}, \groups{})$ from $t_n$ to $t_m$. 

\paragraph{Scope}

Recall from section \vref{sec:define-the-scope} that the temporal scope of this operation is $\interval{t_n}{t_m}$, and the spatial scope is the group itself and the parent feature.

In the premise of the rule, only $\var{groupID}$ and $\var{parentFeatureID}$ are looked up in the temporal feature model. Consequently, the premise stays within the spatial scope of the rule. In the conclusion of the rule, $\features{}$ are assigned to and looked up at $\var{parentFeatureID}$, and $\groups{}$ at $\var{groupID}$. The helper functions $\var{addChildGroup}$ (figure \vref{fun:add-child-group}) and $\var{setGroupAttributes}$ (figure \vref{fun:set-group-attributes}) do not take the temporal feature model as argument, and so only affects the parent feature and the group itself, respectively.

As for the temporal scope, the only interval looked up in the rule is $\interval{t_n}{t_m}$. Hence the rule operates only within the defined temporal scope.
\\

\begin{lemma}
   The \rulefont{Add-Group} rule operates strictly within the scope of the \textbf{addGroup} operation.
   \label{lemma:add-group-scope}
\end{lemma}

\paragraph{Preserving well-formedness}

If the \rulefont{Add-Group} rule is applied, the resulting temporal feature model must be well-formed according to to the well-formedness rules \wf{1\textendash 9}. 

The rule does not change the root feature's existence or type, so it does not violate \wf{1} or \wf{2}. The $\names{}$ map is left unchanged, and the only change made to a feature is to the parent feature, adding $\var{groupID}$ to the set of child groups at $\interval{t_n}{t_m}$. The only feature modified is the parent feature, and only in its child groups map $F_c$. Since $\var{parentFeatureID}$ is assigned to the group's parent feature table $F_p$ at the same key $\interval{t_n}{t_m}$, \wf{3} holds.

Given that \wf{8} holds in the original model, and as the rule premise makes certain that the group does not already exist during the interval $\interval{t_n}{t_m}$, the group does not have any types, parent features, or child features during the interval. When the rule is applied, the group is given exactly one type and parent feature, and $\interval{t_n}{t_m}$ is added to its existence set. Thus \wf{4}, \wf{6}, and \wf{8} hold. 

As for \wf{5}, this requirement holds trivially given that it holds in the original model. No feature is added or removed from any group in the \rulefont{Add-Group} rule, so this condition is not affected and thus still holds.

Similarly, \wf{7} will hold in the altered model given that it holds in the original one, since the new group does not contain any features during the temporal scope. For the same reason, the rule does not create a cycle, and so \wf{9} is true for the altered model.
\\

\begin{lemma}
   The \rulefont{Add-Group} rule preserves well-formedness of the temporal feature model.
   \label{lemma:add-group-well-formed}
\end{lemma}

\paragraph{Correctness of model modification}

The operation is intended to add the group with ID $\var{groupID}$ to the temporal feature model during the interval $\interval{t_n}{t_m}$. Since groups have no names, this operation should not affect the $\names{}$ map. Indeed, the rule reflects this, as the map is not changed in the transition.

However, the operation does naturally add information to the $\groups{}$ map, assigning
\begin{equation*}
   \var{setGroupAttributes}( \lookup{\groups{}}{\var{groupID}}, \var{type}, 
   \var{parentFeatureID} )
\end{equation*}
to $\lookup{\groups}{\var{groupID}}$.

Looking up the added group's ID in the modified model should return the information we put in the operation, and given $\lookup{\groups}{\var{groupID}} = \group$, the following statements hold:
\begin{align}
   \interval{t_n}{t_m} &\in G_e && \text{the group exists} \tag{1} \label{add-group1}\\
   \lookup{G_t}{\interval{t_n}{t_m}} &= \set{\var{type}}  && \text{the group has the expected type} \tag{2} \label{add-group2}\\
   \lookup{G_p}{\interval{t_n}{t_m}} &= \set{\var{parentGroupID}}  && \text{the group has the expected parent feature} \tag{3} \label{add-group3}\\
   \lookup{G_c}{\interval{t_n}{t_m}} &= \emptyset && \text{the group has no children} \tag{4} \label{add-group4}
\end{align}

Statement (\ref{add-group1}) holds due to the line $G_e \cup \interval{t_n}{t_m}$ in $\var{setGroupAttributes}$ (figure \vref{fun:set-group-attributes}). Given the semantics of assignment, also statement (\ref{add-group2}) and (\ref{add-group3}) hold, as the type and parent feature ID are assigned to $\lookup{G_t}{\interval{t_n}{t_m}}$ and $\lookup{G_p}{\interval{t_n}{t_m}}$ respectively in $\var{setGroupAttributes}$. Given that \wf{8} is true for the original model, and since $\var{setGroupAttributes}$ does not modify $G_c$, statement (\ref{add-group4}) is also true.

Furthermore, we would expect the group to be listed as a child group of the parent feature in the modified model, so given that $\lookup{\features{}}{\var{parentFeatureID}} = \feature$, then
\begin{equation*}
   \var{groupID} \in \bigcup \lookup{F_p}{\interval{t_n}{t_m}}
\end{equation*}
In the \rulefont{Add-Group} rule, 
\[\var{addChildGroup}(\lookup{\features}{\var{parentFeatureID}}, \interval{t_n}{t_m}, \var{groupID})\]
is assigned to $\lookup{\features}{\var{parentFeatureID}}$. The function $\var{addChildGroup}$ (figure \vref{fun:add-child-group}) adds $\var{groupID}$ to the set of child features at key $\interval{t_n}{t_m}$, so according to the semantics of $\addassign$, it is indeed true that the group is in the parent feature's set of child group in the temporal scope.
\\

\begin{lemma}
   The \rulefont{Add-Group} rule updates the temporal feature model according to the semantics of the \textbf{addGroup} operation.
   \label{lemma:add-group-mod}
\end{lemma}



\subsection{Soundness of the Remove feature rule}
\label{sub:soundness-of-the-remove-feature-rule}

See figure \vref{rule:remove-feature} for the \rulefont{Remove-Feature} rule. 
Let 
\begin{equation*}
   \begin{split}
      \textbf{removeFeature}(\var{featureID})\text{ at } t_n
      \shove \\
   (\names{}, \features{}, \groups{})
\end{split}
\end{equation*}
be the initial state. Recall that this operation removes the feature with ID \var{featureID} from the interval-based feature model $(\names{}, \features{}, \groups{})$ at $t_n$. Furthermore, let $\lookup{\features}{\var{featureID}} = \feature$, and $\interval{t_i}{t_j} \in F_e$, with $t_i \leq t_n < t_j$. This means that the original plan added the feature at time $t_i$ and removed it at $t_j$, with the possibility that $t_j = \forever$. In the latter case, there was no plan to remove the feature originally.

\paragraph{Scope}
As defined in section \vref{sec:define-the-scope}, the \textbf{removeFeature} operation's temporal scope is $\interval{t_n}{t_k}$, where $t_k$ is the time point in which the feature was originally planned to be removed. We can see from the description above that $t_k = t_j$; the end point in the feature's existence set containing $t_n$. We then have that the scope is defined as $\interval{t_n}{t_j}$. In the rule, we find the interval $\interval{t_i}{t_j}$ by looking up
\begin{equation*}
   \containing{F_e}{t_n} = \set{\interval{t_{e_1}}{t_{e_2}}}
\end{equation*}
   According to the semantics of $\containing{\map{IS}}{t_n}$, it is true then that $\interval{t_i}{t_j} = \interval{t_{e_1}}{t_{e_2}}$, and so the temporal scope of the rule is $\interval{t_n}{t_{e_2}} = \interval{t_n}{t_j}$. Clearly, all time points looked up in the premise of the rule are contained within this interval, but the conclusion requires further examination. The $\names{}$ map is assigned $\var{clampInterval}(\lookup{\names}{\var{name}}, t_n)$ at key $\var{name}$. In $\var{clampInterval}$ (figure \vref{fun:clamp-interval}), the interval $\interval{t_{n_1}}{t_{n_2}}$ containing $t_n$ in $\lookup{\names}{\var{name}}$ is looked up and shortened to end at $t_n$ instead of $t_{n_2}$. This modification stays within the scope of the interval-based feature model, since the interval affected here is $\interval{t_n}{t_{n_2}}$, and necessarily, $t_{n_2} \leq t_j$, since the feature cannot possibly have a name after it is removed according to \wf{8}. 

   The $\features{}$ map is modified at key $\var{featureID}$ by assigning 
   $$\var{clampFeature}(\lookup{\features}{\var{featureID}}, t_n)$$
   . In $\var{clampFeature}$ (figure \vref{fun:clamp-feature}), the intervals of the feature's name, type, and parent are clamped to end at $t_n$. These modifications, too, stay within the temporal scope, for the reason explained in the above paragraph. The existence interval is clamped in a similar way, and so stays within the temporal scope as well.

   Also, the $\groups{}$ map is assigned $$\var{removeFeatureAt}(\lookup{\groups}{\var{parentGroupID}}, \var{featureID}, t_n)$$ at key $\var{parentGroupID}$. This helper function (figure \vref{fun:remove-feature-at}) modifies the parent group's set of subfeatures by calling $$\var{clampIntervalValue}(G_c, t_c, \var{featureID})$$, which behaves similarly to $\var{clampInterval}$ by clamping the interval containing $t_n$. The difference is that it removes only $\var{featureID}$ from the set of subfeatures, and adds the feature to the set of subfeatures at the shortened interval. We conclude that this modification, too, happens within the temporal scope of the operation, as looking up any time point outside of the temporal scope will return the same results as the original plan.

   Recall that the spatial scope of the rule is the feature itself, its parent group, and its subgroups. The premise
\begin{equation*}
   \lookup{F_c}{\interval{t_n}{t_{e_2}}} = \emptyset
\end{equation*}
   ensures that the operation is not applied unless the feature's set of subgroups is empty. The only features and groups looked up is the feature itself and its parent group. Thus, the rule stays within the spatial scope.
\\

\begin{lemma}
   The $\rulefont{Remove-Feature}$ rule operates strictly within the scope of the \textbf{removeFeature} operation.
   \label{lemma:remove-feature-scope}
\end{lemma}

\paragraph{Preserving well-formedness}

The \rulefont{Remove-Feature} rule contains the premise
\begin{equation*}
   \lookup{F_p}{\interval{t_n}{t_{e_2}}} = \set{\var{parentGroupID}}
\end{equation*}
, ensuring that the feature has \emph{exactly} one parent group during the temporal scope of the rule. Under the assumption that \wf{1} holds in the original model, the feature being removed cannot be the root feature, since the root has no parent group. Furthermore, it means that the feature does not move during the temporal scope, which would be a conflict. Therefore, both \wf{1} and \wf{2} hold in the modified model.

For any time point $t_n$ in the temporal scope of the rule, $t_n \notinn F_e$ due to the semantics of $\var{clampFeature}$ (figure \vref{fun:clamp-feature}), so \wf{3} holds trivially.  The only change made to a group is by the function $\var{removeFeatureAt}$ (figure \vref{fun:remove-feature-at}), which removes the feature from the parent group's map of subgroups during $\interval{t_n}{t_j}$. Hence \wf{5} holds, and since that function does not modify the types map $G_t$ of the group, \wf{4} holds given that it is true for the original model. 

The premise $\lookup{F_c}{\interval{t_n}{t_{e_2}}} = \emptyset$ ensures that the feature to be removed does not have any subgroups during the temporal scope, so no group is left without a parent in the updated model. Thus \wf{6} holds. 

Suppose that the parent group has the type $\xortype{}$ or $\ortype{}$ at some point during the temporal scope. In the original model, no child feature of the group has type $\mandatory{}$ due to the assumption that \wf{7} is true. The \rulefont{Remove-Feature} rule does not add any features or change a feature type, so this requirement still holds for the modified model. 

After applying the rule, we have that $\interval{t_n}{t_j} \notinnr F_e$, which means that the feature does not exist during the temporal scope of the operation. To fulfil \wf{8}, we must furthermore have that $\lookup{F_n}{\interval{t_n}{t_j}} = \lookup{F_t}{\interval{t_n}{t_j}}= \lookup{F_p}{\interval{t_n}{t_j}} = \lookup{F_c}{\interval{t_n}{t_j}} = \emptyset$, and that $\var{featureID} \notin \lookup{\lookup{\names}{\var{name}}}{\interval{t_n}{t_j}}$. The former statement holds due to the semantics of $\var{clampFeature}$ and $\var{clampInterval}$; the feature's attributes are all clamped to end at the time of removal, and the premises on the form $\lookup{F_x}{\interval{t_n}{t_{e_2}}} = \set{v}$ ensure that no changes are made to those attributes during the temporal scope. $\lookup{F_c}{\interval{t_n}{t_j}} = \emptyset$ is a premise in the rule (since $t_j = t_{e_2}$). As for the $\names$ map, the mapping from $\var{name}$ to $\intervalmapping{t_i}{t_j}{\var{featureID}}$ is replaced by $\intervalmapping{t_i}{t_n}{\var{featureID}}$ in the function $\var{clampInterval}(\lookup{\names}{\var{name}}, t_n)$. Hence \wf{8} is true for the altered interval-based feature model.

Under the assumption that no cycles exist in the original model, removing a feature does not create a new one, so \wf{9} is holds for the modified model as well.
\\

\begin{lemma}
   The \rulefont{Remove-Feature} rule preserves well-formedness of the interval-based feature model. 
   \label{lemma:remove-feature-well-formed}
\end{lemma}

\paragraph{Correctness of model modification}
The semantics of the \textbf{removeFeature} operation is that applying it should remove the feature from the plan from $t_n$ until the point at which is was originally planned to be removed. Then if $\lookup{\features}{\var{featureID}} = \feature{}$, and $\containing{F_e}{t_n} = \interval{t_i}{t_j}$, then
\begin{align}
   \interval{t_n}{t_m} &\notinnr F_e && \text{the feature does not exist} \tag{1} \label{remove-feature1}\\
   \lookup{F_n}{\interval{t_n}{t_m}} &= \emptyset && \text{the feature has no name} \tag{2} \label{remove-feature2}\\
   \lookup{F_t}{\interval{t_n}{t_m}} &= \emptyset  && \text{the feature has no type} \tag{3} \label{remove-feature3}\\
   \lookup{F_p}{\interval{t_n}{t_m}} &= \emptyset  && \text{the feature has no parent group} \tag{4} \label{remove-feature4}\\
   \lookup{F_c}{\interval{t_n}{t_m}} &= \emptyset && \text{the feature has no subgroups} \tag{5} \label{remove-feature5}
\end{align}
Since we established $\interval{t_n}{t_m} \notinnr F_e$ in the above paragraph, these
statements follow directly from lemma \vref{lemma:remove-feature-well-formed} and \wf{8}. It further follows that no name is associated with $\var{featureID}$ in the $\names{}$ map, and that no group in the $\groups{}$ map has the feature listed as a subfeature.
\\

\begin{lemma}
   The \rulefont{Remove-Feature} rule updates the interval-based feature model according to the semantics of the \textbf{removeFeature} operation.
   \label{lemma:remove-feature-mod}
\end{lemma}


\subsection{Soundness of the Remove Group Rule}
\label{sub:soundness-of-the-remove-group-rule}
This proof is analogous to the one for the \rulefont{Remove-Feature} rule.
See Figure~\vref{rule:remove-group} for the \rulefont{Remove-Group} rule. Let
\[
      \textbf{removeGroup}(\var{groupID}) \text{ at } t_n \shove (\names, \features, \groups)
\]
be the initial state, and
\[
   (\names', \features', \groups')
\]
be the result state after applying the \rulefont{Remove-Group} rule. Recall that this operation removes the group with ID $\var{groupID}$ from the interval-based feature model $(\names, \features, \groups)$ at $t_n$. Furthermore, let $\lookup{\groups}{\var{groupID}} = \group{}$, and $\interval{t_i}{t_j} \in G_e$, with $t_i \leq t_n < t_j$. This means that the original plan added the group at time $t_i$ and removed it at $t_j$, with the possibility that $t_j = \forever$. In the latter case, there was no plan to remove the group originally.

\paragraph{Scope}

As defined in Section~\vref{sec:scope}, the \textbf{removeGroup} operation's temporal scope is $\interval{t_n}{t_k}$, where $t_k$ is the time point in which the group was originally planned to be removed. We can see from the description above that $t_k = t_j$; the end point in the group's existence set containing $t_n$. We then have that the scope is defined as $\interval{t_n}{t_j}$. In the rule, we find the interval $\interval{t_i}{t_j}$ by looking up
\[
   \containing{G_e}{t_n} = \set{\interval{t_{e_1}}{t_{e_2}}}.
\]
According to the semantics of $\containing{\map{IS}}{t_n}$, it is true then that $\interval{t_i}{t_j} = \interval{t_{e_1}}{t_{e_2}}$, and so the temporal scope of the rule is $\interval{t_n}{t_{e_2}} = \interval{t_n}{t_j}$. Clearly, all time points looked up in the premise of the rule are contained within this interval, but the conclusion requires further examination. The $\names{}$ map is untouched and thus outside the scope. 

   The $\groups{}$ map is modified at key $\var{groupID}$ by assigning 
   $$\var{clampGroup}(\lookup{\groups}{\var{groupID}}, t_n)$$
   . In $\var{clampGroup}$ (Figure~\vref{fun:clamp-group}), the intervals of the group's type and parent feature are clamped to end at $t_n$. These modifications stay within the temporal scope, as $\var{clampInterval}(\map{MAP}, t_c)$ clamps the mapping with an interval key containing $t_c$ to end at $t_c$. Due to \wf{8}, it is impossible that the group has a type or parent feature after $t_j$, which is the time point when the group was originally planned to be removed. Furthermore, the premise of the rule requires that $\lookup{G_t}{\interval{t_n}{t_j}} = \set{\var{type}}$ and $\lookup{G_p}{\interval{t_n}{t_j}} = \set{\var{parentFeatureID}}$, meaning that the group does not change its type or move during the temporal scope. Thus there is only one key in each of the group's type and parent feature maps containing $t_n$, and so the changed interval for these maps is $\interval{t_n}{t_j}$; the temporal scope. The existence interval is clamped in a similar way, and so stays within the temporal scope as well.

   Also, the $\features{}$ map is assigned $$\var{removeGroupAt}(\lookup{\features}{\var{parentFeatureID}}, \var{groupID}, t_n)$$ at key $\var{parentFeatureID}$. This helper function (Figure~\vref{fun:remove-group-at}) modifies the parent feature's set of subgroups by calling $$\var{clampIntervalValue}(F_c, t_c, \var{groupID})$$, which behaves similarly to $\var{clampInterval}$ by clamping the interval containing $t_n$. The difference is that it removes only $\var{groupID}$ from the set of child groups, and adds the group to the set of subgroups at the shortened interval. We conclude that this modification, too, happens within the temporal scope of the operation, as looking up any time point outside of the temporal scope will return the same results as the original plan.

   Recall that the spatial scope of the rule is the group itself, its parent feature, and its subfeatures. The premise
\begin{equation*}
   \lookup{G_c}{\interval{t_n}{t_{e_2}}} = \emptyset
\end{equation*}
   ensures that the operation is not applied unless the group's set of subfeatures is empty. The only features and groups looked up is the group itself and its parent feature. Thus, the rule stays within the spatial scope.
 \\

\begin{lemma}
   The $\rulefont{Remove-Group}$ rule operates strictly within the scope of the \textbf{removeGroup} operation.
   \label{lemma:remove-group-scope}
\end{lemma}

\paragraph{Preserving well-formedness}

Let
\begin{align*}
   \lookup{\groups}{\var{groupID}} &= \group \\
   \lookup{\groups'}{\var{groupID}} &= \groupvar{G'} \\
   \lookup{\features}{\var{parentFeatureID}} &= \feature \\
   \lookup{\features'}{\var{parentFeatureID}} &= \featurevar{F'}
\end{align*}

The \rulefont{Remove-Group} rule does not alter any feature's \textemdash{} in particular the root feature's \textemdash{} existence set or types map, and so \wf{1\textendash 2} hold for the modified model. It does however modify the subgroup map $F_c$, applying $\var{removeGroupAt}$ to the parent feature, the group's ID, and the removal time point. As previously argued, this function makes sure that the group ID is not in $\bigcup \lookup{F'_c}{\interval{t_n}{t_j}}$ \textemdash{} the parent feature's modified set of subgroups \textemdash{} so \wf{3} holds.

Due to the semantics of $\var{clampGroup}$ and $\var{clampIntervalSet}$, no time points in the temporal scope are contained in an interval in the modified group's existence set ($\interval{t_n}{t_j} \notinnr G'_e$), so \wf{4}, \wf{6} and \wf{7} hold trivially. 

Since a premise of the rule is
   \[
      \lookup{G_c}{\interval{t_n}{t_{e_2}}} = \emptyset
   \]
   , the group does not have any subfeatures during the temporal scope in the original interval-based feature model. Due to the assumption that \wf{5} is true for the original model, no feature has $\var{groupID}$ listed as its parent group, so no feature is left without a parent group when the group is removed from the temporal scope. It follows that \wf{5} holds for the updated interval-based feature model as well. 

   As previously mentioned, $\interval{t_n}{t_j} \notinnr G'_e$, meaning that the group does not exist during the temporal scope in the modified model. For \wf{8} to hold for the updated model, we must then also have $\lookup{G'_t}{\interval{t_n}{t_j}} = \lookup{G'_p}{\interval{t_n}{t_j}} = \lookup{G'_c}{\interval{t_n}{t_j}} = \emptyset$. Recalling that $t_{e_2} = t_j$, then by definition of $\var{clampGroup}$ and the premise $\lookup{G_t}{\interval{t_n}{t_{e_2}}} = \set{\var{type}}$ in \rulefont{Add-Group}, we have that
\begin{align*}
   G'_t &= \var{clampInterval}\left(G_t, t_n\right) \\
        &= \var{clampInterval}\left(G''_t \cup \intervalmapping{t_{t_1}}{t_j}{\var{type}}, t_n\right) \\
        &= G''_t \cup \intervalmapping{t_{t_1}}{t_n}{\var{type}}
\end{align*}
Clearly, $\lookup{G''_t}{\interval{t_n}{t_j}} = \emptyset$.  Furthermore, since $\interval{t_{t_1}}{t_n}$ does not overlap $\interval{t_n}{t_j}$, $\lookup{G'_t}{\interval{t_n}{t_j}} = \emptyset$. An analogous argument can be made for $\lookup{G'_p}{\interval{t_n}{t_j}} = \emptyset$. From the definition of $\var{clampGroup}$, $G_c = G'_c$, so by the premise $\lookup{G_c}{\interval{t_n}{t_j}} = \emptyset$ in the rule, $\lookup{G_c'}{\interval{t_n}{t_j}} = \emptyset$. Consequently \wf{8} holds for the altered interval-based feature model.

Given that no cycles exist in the original model, removing a group does not create a new one, so \wf{9} holds.
\\

\begin{lemma}
   The \rulefont{Remove-Group} rule preserves well-formedness of the interval-based feature model.
   \label{lemma:remove-group-well-formed}
\end{lemma}

\paragraph{Correctness of model modification}
% Vis: NAMES=NAMES' er uendret
% Vis: \var{groupID} \notin \bicup \lookup{F'_c}{\interval{t_n}{t_j}}
% Vis: \interval{t_n}{t_j} \notinnr G'_e
% \lookup{G_t}{\interval{t_n}{t_j}} = \lookup{G_p}{\interval{t_n}{t_j}} = \lookup{G_c}{\interval{t_n}{t_j}}
The semantics of the \textbf{removeGroup} operation dictate that the group should not exist, have a type, a parent, or subfeatures after being removed. A proof for this can be found in the previous paragraph. Moreover, the parent feature's map of child features should not contain $\var{groupID}$ during the temporal scope. This is also proven in the previous paragraph. The $\names{}$ map should not be modified. It is clear from the rule that $\names{} = \names{}'$, so this condition, too, is true.
\\

\begin{lemma}
   The \rulefont{Remove-Group} rule updates the interval-based feature model according to the semantics of the \textbf{removeGroup} operation.
   \label{lemma:remove-group-mod}
\end{lemma}


\subsection{Soundness of the Move feature rule}
\label{sub:soundness-of-the-move-reature-rule}

See figure \vref{rule:move-feature} for the \rulefont{Move-Feature} rule. Let 
\begin{equation*}
   \begin{split}
      \textbf{moveFeature}(\var{featureID}, \var{newParentID}) \text{ at } t_n \shove \\
      (\names, \features, \groups)
   \end{split}
\end{equation*}
be the initial state, and
\[
   (\names', \features', \groups')
\]
be the result state after applying the \rulefont{Move-Feature} rule. Recall that this operation moves the feature with ID $\var{featureID}$ to the group with ID $\var{newParentID}$.

\paragraph{Scope}
\label{par:scope}
Recall that the temporal scope of the move-feature rule is $\interval{t_n}{t_k}$ (section \vref{sec:define-the-scope}), where $t_k$ is the time point at which the feature is originally planned to be moved or is removed. In the rule, this scope is identified by 
   \[
      \containing{F_p}{t_n} = \set{\interval{t_{p_1}}{t_{p_2}}}
   \]
   Here, the time point $t_n$ for moving the feature is looked up in the feature's parent map's set of interval keys, and the expected result is $\set{\interval{t_{p_1}}{t_{p_2}}}$. This means that there is a mapping $\intervalmapping{t_{p_1}}{t_{p_2}}{\var{parentGroupID}}$ in $F_p$, with $\var{parentGroupID}$ being the ID of the feature's parent group at time $t_n$, and this group stops being the feature's parent at $t_{p_2}$. Thus the temporal scope of this operation is $\interval{t_n}{t_{p_2}}$. The only interval looked up or assigned to in the rule is $\interval{t_n}{t_{p_2}}$, but it is necessary to also look at the cycle detection algorithm in section \vref{sub:move-algorithm}, since this is also referenced in the rule by $\neg \var{createsCycle}$. Here, $t_{p_2}$ is called $t_e$, and the algorithm states that it only looks at time points between $t_n$ and $t_e$. Thus the rule operates strictly within the temporal scope of the \textbf{moveFeature} operation.

   The spatial scope for this operation is defined as the \emph{ancestors which the feature and the target group do not have in common}. In other words, the \emph{new} ancestors of the feature after applying the rule. In the rule itself, only the feature with ID $\var{featureID}$ and its new parent group with ID $\var{newParentID}$ are looked up. However, the cycle detection algorithm must also be considered. Here, the ancestors of both the feature and the group at $t_n$ are looked up, the first ancestor they have in common identified, and the new ancestors are collected into a list. If one of them is moved before $t_e$, the list is updated. Hence the algorithm's spatial scope is indeed the feature's new ancestors and the feature itself, and so the rule operates within the defined spatial scope.
\\

\begin{lemma}
   The \rulefont{Move-Feature} rule operates strictly within the scope of the \textbf{moveFeature} operation.
   \label{lemma:move-feature-scope}
\end{lemma}

\paragraph{Preserving well-formedness}
Since the rule verifies that the feature has a parent group, the feature being moved is not the root. Thus \wf{1} and \wf{2} hold. The rule does not update the name, type or subgroups of the feature, so \wf{3} is true for the updated model. Nor does it modify the target group's type or parent feature, so \wf{4}, \wf{6}, and \wf{7} also hold.

The modification made to $\lookup{\features}{\var{featureID}}$ is to the parent group map $F_p$ by 
\[
   F_p' = \lookup{\var{clampInterval}(F_p, \, t_n)}{\interval{t_n}{t_{p_2}}} \assign \var{newParentID}
\]
As discussed in earlier sections (e.g. \vref{sub:soundness-of-the-remove-feature-rule}), $\var{clampInterval}$ replaces a mapping $\intervalmapping{t_i}{t_j}{v}$ by $\intervalmapping{t_i}{t_n}{v}$, with $t_n \leq t_j$. Thus $\var{clampInterval}(F_p, \, t_n)$ has no parent group during the temporal scope. The subsequent assignment of $\var{newParentID}$ to $\interval{t_n}{t_{p_2}}$ ensures that the feature has exactly one parent group during the temporal scope. This relation is reflected in the $\groups{}'$ map, with
\begin{align*}
   \groups' &= \\
   \lookup{&\big(\lookup{\groups{}}{\var{oldParentID}} \\
   \assign & \var{removeFeatureAt}\left(\lookup{\groups{}}{\var{oldParentID}}, \var{featureID}, t_n\right)\big)}{\var{newParentID}} \\
   \assign& \var{addChildFeature}(\lookup{\groups{}}{\var{newParentID}}, \interval{t_n}{t_{p_2}}, \var{featureID})
\end{align*}
The feature is added to the target group by $\var{addChildFeature}$ during the interval $\interval{t_n}{t_{p_2}}$, and removed from the original parent group by $\var{removeFeatureAt}$. Consequently \wf{5} holds for the updated model. 

As the rule does not alter the feature's existence set, \wf{8} is preserved. 

The intention of the cycle detection algorithm in section \vref{sub:move-algorithm} is to uphold \wf{9}. Given the assumption that the original temporal feature model contains no cycles, if the altered model contains a cycle then the $\textbf{moveFeature}$ operation introduced it, and the feature being moved must be part of the cycle. This could only happen if the feature became part of its own subtree during the temporal scope, which means that at some point, the feature occurs in its own list of ancestors. The algorithm looks at the feature's \emph{new} ancestors, meaning the ancestors that the feature does not have in the original plan, but does in the new one. It then checks that none of those ancestors are moved to the feature's subtree. Thus the rule preserves \wf 9.
\\

\begin{lemma}
   The \rulefont{Move-Feature} rule preserves well-formedness of the temporal feature model. 
   \label{lemma:move-feature-well-formed}
\end{lemma}

\paragraph{Correctness of model modification}
The operation is intended to move the feature with ID $\var{featureID}$ to the group with ID $\var{newParentID}$ during the temporal scope $\interval{t_n}{t_{p_2}}$. After applying the \rulefont{Move-Feature} rule, the only differences between the original and modified temporal feature model should be
\begin{enumerate}[(i)]
   \item The feature's parent group should be $\var{newParentID}$ during the temporal scope \label{move-feature-parentgroup}
   \item The feature should not appear in the original parent group's set of subfeatures during the temporal scope \label{move-feature-oldparent}
   \item The feature should appear in the new parent group's set of subfeatures \label{move-feature-newparent}
\end{enumerate}

Given the modified map of parent groups $F'_p$ and the original map $F_p$, we have that
\[
   F'_p = \lookup{\var{clampInterval}(F_p, t_n)}{\interval{t_n}{t_{p_2}}} \assign \var{newParentID}
\]
This statement assigns $\var{newParentID}$ to the temporal scope $\interval{t_n}{t_{p_2}}$ after applying $\var{clampInterval}(F_p, t_n)$, meaning that the original parent mapping is shortened to end at $t_n$, and a new mapping $\intervalmapping{t_n}{t_{p_2}}{\var{newParentID}}$ is inserted. By semantics of assignment, it is clear that $\lookup{F'_p}{\interval{t_n}{t_{p_2}}} = \set{\var{newParentID}}$, which is the desired result and fulfils \ref{move-feature-parentgroup}.



\begin{lemma}
   The \rulefont{Move-Feature} rule updates the temporal feature model according to the semantics of the \textbf{moveFeature} operation.
   \label{lemma:move-feature-mod}
\end{lemma}



\subsection{Soundness of the Move group rule}
\label{sub:soundness-of-the-move-group-rule}

See figure \vref{rule:move-group} for the \rulefont{Move-Group} rule. Let 
\begin{equation*}
   \begin{split}
      \textbf{moveGroup}(\var{groupID}, \var{newParentID}) \text{ at } t_n \shove \\
      (\names, \features, \groups)
   \end{split}
\end{equation*}
be the initial state, and
\[
   (\names', \features', \groups')
\]
be the result state after applying the \rulefont{Move-Group} rule. Recall that this operation moves the group with ID $\var{groupID}$ to the feature with ID $\var{newParentID}$.

\paragraph{Scope}
Recall that the temporal scope of the \rulefont{Move-Group} rule is $\interval{t_n}{t_k}$ (section \vref{sec:define-the-scope}), where $t_k$ is the time point at which the group is originally planned to be moved or is removed. In the rule, this scope is identified by 
   \[
      \containing{G_p}{t_n} = \set{\interval{t_{p_1}}{t_{p_2}}}
   \]
   Here, the time point $t_n$ for moving the group is looked up in the group's parent map's set of interval keys, and the expected result is $\set{\interval{t_{p_1}}{t_{p_2}}}$. This means that there is a mapping $\intervalmapping{t_{p_1}}{t_{p_2}}{\var{parentFeatureID}}$ in $F_p$, with $\var{parentFeatureID}$ being the ID of the group's parent feature at time $t_n$, and this feature stops being the group's parent at $t_{p_2}$. Thus the temporal scope of this operation is $\interval{t_n}{t_{p_2}}$. The only interval looked up or assigned to in the rule is $\interval{t_n}{t_{p_2}}$, but it is necessary to also look at the cycle detection algorithm in section \vref{sub:move-algorithm}, since this is also referenced in the rule by $\neg \var{createsCycle}$. Here, $t_{p_2}$ is called $t_e$, and the algorithm states that it only looks at time points between $t_n$ and $t_e$. Thus the rule operates strictly within the temporal scope of the \textbf{moveGroup} operation.

   The spatial scope for this operation is defined as the \emph{ancestors which the group and the target feature do not have in common}. In other words, the \emph{new} ancestors of the group after applying the rule. In the rule itself, only the group with ID $\var{groupID}$ and its new parent feature with ID $\var{newParentID}$ are looked up. However, the cycle detection algorithm must also be considered. Here, the ancestors of both the group and the feature at $t_n$ are looked up, the first ancestor they have in common identified, and the new ancestors are collected into a list. If one of them is moved before $t_e$, the list is updated. Hence the algorithm's spatial scope is indeed the group's new ancestors and the group itself, and so the rule operates within the defined spatial scope.
\\

\begin{lemma}
   The \rulefont{Move-Group} rule operates strictly within the scope of the \textbf{moveGroup} operation.
   \label{lemma:move-group-scope}
\end{lemma}

\paragraph{Preserving well-formedness}
Let \var{oldParentID} be the ID of the group's parent feature in the original plan, and let
\begin{align*}
   \lookup{\features'}{\var{oldParentID}} &= \featurevar{OP} \\
   \lookup{\features'}{\var{newParentID}} &= \featurevar{NP} \\
   \lookup{\groups'}{\var{groupID}} &= \groupvar{G'}
\end{align*}

Since the \rulefont{Move-Group} rule does not remove or change the type of a feature, \wf{1} and \wf{2} hold. The modification made to the $\features{}$ map is 
\begin{align*}
   \lookup{\big(\lookup{\features{}&}{\var{oldParentID}} \\
   \assign  \var{removeGroupAt}& (\lookup{\features{}}{\var{oldParentID}}, \interval{t_n}{t_{p_2}}, \var{groupID})\big)}{\var{newParentID}} \\
   \assign  \var{addChildGroup}& \left(\lookup{\features{}}{\var{newParentID}}, \var{groupID}, t_n\right)
\end{align*}
This change modifies only the subgroup maps of the original and new parent features of the group. In the modified model, for any time point $t_i$ in the temporal scope, $\var{groupID} \notin \lookup{OP_c}{t_i}$, and $\var{groupID} \in \lookup{NP}{t_i}$. Furthermore, the $\groups{}$ map is changed by
\begin{align*}
   \lookup{\groups{}}{\var{groupID}} \assign &(G_e,\, G_n,\, G_t,\, \\
                                             & \lookup{\var{clampInterval}(G_p, t_n)}{\interval{t_n}{t_{p_2}}} \assign \var{newParentID},\, G_c)
\end{align*}
meaning that $\lookup{G'_p}{t_i} = \set{\var{newParentID}}$. Hence \wf{3} and \wf{6} hold. 

As the group's types and subfeatures map are not modified, \wf{4\textendash 5} and \wf{7} are true for the modified model.  Similarly, since the rule does not alter the group's existence set, \wf{8} is preserved. 

The intention of the cycle detection algorithm in section \vref{sub:move-algorithm} is to uphold \wf{9}. Given the assumption that the original temporal feature model contains no cycles, if the altered model contains a cycle then the $\textbf{moveGroup}$ operation introduced it, and the group being moved must be part of the cycle. This could only happen if the group became part of its own subtree during the temporal scope, which means that at some point, the group occurs in its own list of ancestors. The algorithm looks at the group's \emph{new} ancestors, meaning the ancestors that the group does not have in the original plan, but does in the new one. It then checks that none of those ancestors are moved to the group's subtree. Thus the rule preserves \wf 9.
\\

\begin{lemma}
   The \rulefont{Move-Group} rule preserves well-formedness of the temporal feature model. 
   \label{lemma:move-group-well-formed}
\end{lemma}

\paragraph{Correctness of model modification}
The operation is intended to move the group with ID $\var{groupID}$ to the feature with ID $\var{newParentID}$ during the temporal scope $\interval{t_n}{t_{p_2}}$. After applying the \rulefont{Move-Group} rule, the only differences between the original and modified temporal feature model should be
\begin{enumerate}[(i)]
   \item The group's parent feature should be $\var{newParentID}$ during the temporal scope \label{move-group-parentfeature}
   \item The group should not appear in the original parent feature's set of subgroups during the temporal scope \label{move-group-oldparent}
   \item The group should appear in the new parent feature's set of subgroups \label{move-group-newparent}
\end{enumerate}

Given the modified map of parent features $G'_p$ and the original map $G_p$, we have that
\[
   G'_p = \lookup{\var{clampInterval}(G_p, t_n)}{\interval{t_n}{t_{p_2}}} \assign \var{newParentID}
\]
This statement assigns $\var{newParentID}$ to the temporal scope $\interval{t_n}{t_{p_2}}$ after applying $\var{clampInterval}(G_p, t_n)$, meaning that the original parent mapping is shortened to end at $t_n$, and a new mapping $\intervalmapping{t_n}{t_{p_2}}{\var{newParentID}}$ is inserted. By semantics of assignment, it is clear that for $t_i$ with $t_n \leq t_i < t_{p_2}$, $\lookup{G'_p}{t_i} = \set{\var{newParentID}}$, which is the desired result and fulfils \ref{move-group-parentfeature}.

By lemma \vref{lemma:move-group-well-formed} and \wf{5}, \ref{move-group-oldparent} and \ref{move-group-newparent} follow from \ref{move-group-parentfeature}. In other words, since the updated temporal feature model is well-formed, and the group's parent feature during the temporal scope is $\var{newParentID}$, the group is not in the original parent feature's set of subgroups during the temporal scope, and is in the new parent feature's set of subgroups.
\\

\begin{lemma}
   The \rulefont{Move-Group} rule updates the temporal feature model according to the semantics of the \textbf{moveGroup} operation.
   \label{lemma:move-group-mod}
\end{lemma}



\section{Soundness of the Change Feature Variation Type Rule} 
\label{sec:soundness-of-the-change-feature-variation-type-rule}

See Figure~\vref{rule:change-feature-varation-type} for the \rulefont{Change-Feature-Variation-Type} rule. Let 
\begin{equation*}
   \begin{split}
      \textbf{changeFeatureVariationType}(\var{featureID}, \var{type}) \text{ at } t_n \shove \\
      (\names, \features, \groups)
   \end{split}
\end{equation*}
be the initial state, and
\[
   (\names', \features', \groups')
\]
be the result state after applying the \rulefont{Change-Feature-Variation-Type} rule. Recall that this operation changes the type of the feature with ID $\var{featureID}$ to $\var{type}$.

\paragraph{Modularity}
Recall that the temporal scope of the \rulefont{Change-Feature-Variation-Type} rule is $\interval{t_n}{t_k}$ (Section~\vref{sec:scope}), where $t_k$ is the time point at which the type is originally planned to be changed or the feature is removed. In the rule, this scope is identified by 
   \[
      \containing{F_t}{t_n} = \set{\interval{t_{t_1}}{t_{t_2}}}
   \]
   Here, the time point $t_n$ for changing the feature type is looked up in the feature's types map's set of interval keys, and the expected result is $\set{\interval{t_{t_1}}{t_{t_2}}}$. This means that there is a mapping $\intervalmapping{t_{t_1}}{t_{t_2}}{\var{oldType}}$ in $F_t$, with $\var{oldType}$ being the type of the feature at time $t_n$, and this stops being the case at $t_{t_2}$. Thus the temporal scope of this operation is $\interval{t_n}{t_{t_2}}$. The only interval looked up or assigned to in the rule is $\interval{t_n}{t_{t_2}}$, so the rule operates strictly within the temporal scope of the operation.

   The spatial scope for this operation is the feature itself and its parent group. Since the feature may move during the temporal scope, there may be several parent groups to consider. These groups and their types are looked up in the premise
\[
   \begin{split}
      \forall \interval{t_{p_1}}{t_{p_2}} \in \overlapping{F_p}{t_n}{t_{t_2}}  \\ % for all parent intervals overlapping the temporal scope
      \forall p \in \lookup{F_p}{\interval{t_{p_1}}{t_{p_2}}}  \\% for all parent groups (always exactly one) in the parent interval
      \forall t \in \, \var{getTypes}\left(\lookup{\groups}{p}, \clamp{\interval{t_{p_1}}{t_{p_2}}}{t_n}{t_{t_2}}\right)  \\
          \big(\var{compatibleTypes}(t, \var{type})\big) 
    \end{split}
\]

Otherwise, the only feature or group looked up or assigned to in the rule is $\lookup{features}{\var{featureID}}$, so the rule stays within the spatial scope.

Based on the above proof, we conclude with the following lemma:
\\

\begin{lemma}
   The \rulefont{Change-Feature-Variation-Type} rule operates strictly within the temporal and spatial scopes of the \textbf{changeFeatureVariationType} operation.
   \label{lemma:change-feature-variation-type-scope}
\end{lemma}

\paragraph{Preserving well-formedness}
Due to the premise $\var{featureID} \neq RootID$, the feature is not the root, so \wf{1\textendash 2} hold trivially. The modification to $F_t$
\[
   \lookup{\var{clampInterval}(F_t, t_n)}{\interval{t_n}{t_{t_2}}} \assign \var{type}
\]
ensures that the feature's original type stops at $t_n$ and the new one lasts for the duration of the temporal scope $\interval{t_n}{t_{t_2}}$. Since the feature has exactly one type during the temporal scope, and no other modifications are made to the feature, \wf{3} is preserved. Because of this, and since the $\groups{}$ map is also left unchanged, \wf{4\textendash 6} and \wf{8\textendash 9} hold. 

As discussed in the \textbf{Scope} paragraph, the premise 
\[
   \begin{split}
      \forall \interval{t_{p_1}}{t_{p_2}} \in \overlapping{F_p}{t_n}{t_{t_2}}  \\ % for all parent intervals overlapping the temporal scope
      \forall p \in \lookup{F_p}{\interval{t_{p_1}}{t_{p_2}}}  \\% for all parent groups (always exactly one) in the parent interval
      \forall t \in \, \var{getTypes}\left(\lookup{\groups}{p}, \clamp{\interval{t_{p_1}}{t_{p_2}}}{t_n}{t_{t_2}}\right)  \\
          \big(\var{compatibleTypes}(t, \var{type})\big) 
    \end{split}
\]
looks up all parent mappings overlapping the temporal scope ($\intervalmapping{t_{p_1}}{t_{p_2}}{p}$), finds the types each parent group has during the scope and \emph{while} it is the parent of the feature, and verifies that those types are compatible. Thus \wf{7} is preserved.

We conclude that the \rulefont{Change-Feature-Variation-Type} rule preserves well-formedness for the interval-based feature model, according to well-formedness rules \wf{1-9}. 
\\

\begin{lemma}
   The \rulefont{Change-Feature-Variation-Type} rule preserves well-formedness of the interval-based feature model. 
   \label{lemma:change-feature-variation-type-well-formed}
\end{lemma}

\paragraph{Correctness of model modification}
The expected result of applying the rule is that $\lookup{\features'}{\var{featureID}} = \featurevar{F'}$ has the type $\var{type}$ during the temporal scope $\interval{t_n}{t_{t_2}}$. Indeed, due to the semantics of $\var{clampInterval}$ and assignment, for any time point $t_i$ such that $t_n \leq t_i < t_{t_2}$,
\[
   \lookup{F'_t}{t_i} = \set{\var{type}}
\]
Since no other part of the interval-based feature model is altered, the rule performs as desired.
\\

\begin{lemma}
   The \rulefont{Change-Feature-Variation-Type} rule updates the interval-based feature model according to the semantics of the \textbf{changeFeatureVariationType} operation.
   \label{lemma:change-feature-variation-type-mod}
\end{lemma}


\section{Soundness of the Change Group Variation Type Rule} 
\label{sec:soundness-of-the-change-group-variation-type-rule}

See Figure~\vref{rule:change-group-varation-type} for the \rulefont{Change-Group-Variation-Type} rule. Let 
\begin{equation*}
   \begin{split}
      \textbf{changeGroupVariationType}(\var{groupID}, \var{type}) \text{ at } t_n \shove \\
      (\names, \features, \groups)
   \end{split}
\end{equation*}
be the initial state, and
\[
   (\names', \features', \groups')
\]
be the result state after applying the \rulefont{Change-Group-Variation-Type} rule. Recall that this operation changes the type of the group with ID $\var{groupID}$ to $\var{type}$.

\paragraph{Modularity}
Recall that the temporal scope of the \rulefont{Change-Group-Variation-Type} rule is $\interval{t_n}{t_k}$ (Section~\vref{sec:scope}), where $t_k$ is the time point at which the type is originally planned to be changed or the group is removed. In the rule, this scope is identified by 
   \[
      \containing{G_t}{t_n} = \set{\interval{t_{t_1}}{t_{t_2}}}
   \]
   Here, the time point $t_n$ for changing the group type is looked up in the group's types map's set of interval keys, and the expected result is $\set{\interval{t_{t_1}}{t_{t_2}}}$. This means that there is a mapping $\intervalmapping{t_{t_1}}{t_{t_2}}{\var{oldType}}$ in $G_t$, with $\var{oldType}$ being the type of the group at time $t_n$, and this stops being the case at $t_{t_2}$. Thus the temporal scope of this operation is $\interval{t_n}{t_{t_2}}$. The only interval looked up or assigned to in the rule is $\interval{t_n}{t_{t_2}}$, so the rule operates strictly within the temporal scope of the operation.

   The spatial scope for this operation is the group itself and its parent feature. The group may have several child features during the temporal scope, which may both move and change their types. These features and their types are looked up in the premise
\[
   \begin{split}
        \forall \interval{t_{c_1}}{t_{c_2}} \in \overlapping{G_c}{t_n}{t_{t_2}}\\ % for all parent intervals overlapping the temporal scope 
        \forall c \in \, \bigcup \lookup{G_c}{\interval{t_{c_1}}{t_{c_2}}}\\ % for all child features in the child interval
        \forall t \in \, \var{getTypes}\left(\lookup{\features}{c}, \clamp{\interval{t_{c_1}}{t_{c_2}}}{t_n}{t_{t_2}}\right) \\ % for all the types of the child features during the temporal scope
        \big(\var{compatibleTypes}(\var{type}, t)\big)
   \end{split}
\]

Otherwise, the only feature or group looked up or assigned to in the rule is $\lookup{groups}{\var{groupID}}$, so the rule stays within the spatial scope.
\\

\begin{lemma}
   The \rulefont{Change-Group-Variation-Type} rule operates strictly within the temporal and spatial scopes of the \textbf{changeGroupVariationType} operation.
   \label{lemma:change-group-variation-type-scope}
\end{lemma}

\paragraph{Preserving well-formedness}

The modification to $G_t$
\[
   \lookup{\var{clampInterval}(G_t, t_n)}{\interval{t_n}{t_{t_2}}} \assign \var{type}
\]
ensures that the group's original type stops at $t_n$ and the new one lasts for the duration of the temporal scope $\interval{t_n}{t_{t_2}}$. Since the group has exactly one type during the temporal scope, \wf{4} holds.

As discussed in the \textbf{Scope} paragraph, the premise 
\[
   \begin{split}
        \forall \interval{t_{c_1}}{t_{c_2}} \in \overlapping{G_c}{t_n}{t_{t_2}}\\ % for all parent intervals overlapping the temporal scope 
        \forall c \in \, \bigcup \lookup{G_c}{\interval{t_{c_1}}{t_{c_2}}}\\ % for all child features in the child interval
        \forall t \in \, \var{getTypes}\left(\lookup{\features}{c}, \clamp{\interval{t_{c_1}}{t_{c_2}}}{t_n}{t_{t_2}}\right) \\ % for all the types of the child features during the temporal scope
        \big(\var{compatibleTypes}(\var{type}, t)\big)
   \end{split}
\]
looks up all child feature mappings overlapping the temporal scope ($\intervalmapping{t_{c_1}}{t_{c_2}}{\set{f_1, f_2, \ldots}}$), finds the types each child feature has during the scope and \emph{while} it is the child feature of the group, and verifies that those types are compatible. Thus \wf{7} is preserved. 
As no changes are made to any other part of the interval-based feature model, the other requirements \wf{1\textendash 3}, \wf{5\textendash 6}, and \wf{8\textendash 9} hold trivially.
\\

\begin{lemma}
   The \rulefont{Change-Group-Variation-Type} rule preserves well-formedness of the interval-based feature model. 
   \label{lemma:change-group-variation-type-well-formed}
\end{lemma}

\paragraph{Correctness of model modification}
The expected result of applying the rule is that $\lookup{\groups'}{\var{groupID}} = \groupvar{G'}$ has the type $\var{type}$ during the temporal scope $\interval{t_n}{t_{t_2}}$. Indeed, due to the semantics of $\var{clampInterval}$ and assignment, for any time point $t_i$ such that $t_n \leq t_i < t_{t_2}$,
\[
   \lookup{G'_t}{t_i} = \set{\var{type}}
\]
Since no other part of the interval-based feature model is altered, the rule performs as desired.
\\

\begin{lemma}
   The \rulefont{Change-Group-Variation-Type} rule updates the interval-based feature model according to the semantics of the \textbf{changeGroupVariationType} operation.
   \label{lemma:change-group-variation-type-mod}
\end{lemma}


\subsection{Soundness of the Change feature name rule} 
\label{sub:soundness-of-the-change-feature-name-rule}

See figure \vref{rule:change-feature-name} for the \rulefont{Change-Feature-Variation-Type} rule. Let 
\begin{equation*}
   \begin{split}
      \textbf{changeFeatureName}(\var{featureID}, \var{name}) \text{ at } t_n \shove \\
      (\names, \features, \groups)
   \end{split}
\end{equation*}
be the initial state, and
\[
   (\names', \features', \groups')
\]
be the result state after applying the \rulefont{Change-Feature-Name} rule. Recall that this operation changes the name of the feature with ID $\var{featureID}$ to $\var{name}$.

\paragraph{Scope}
Recall that the temporal scope of the \rulefont{Change-Feature-Name} rule is $\interval{t_n}{t_k}$ (section \vref{sec:define-the-scope}), where $t_k$ is the time point at which the name is originally planned to be changed or the feature is removed. In the rule, this scope is identified by 
   \[
      \containing{F_n}{t_n} = \set{\interval{t_{n_1}}{t_{n_2}}}
   \]
   Here, the time point $t_n$ for changing the name is looked up in the feature's names map's set of interval keys, and the expected result is $\set{\interval{t_{n_1}}{t_{n_2}}}$. This means that there is a mapping $\intervalmapping{t_{n_1}}{t_{n_2}}{\var{oldName}}$ in $F_n$, with $\var{oldName}$ being the name of the feature at time $t_n$, and this stops being the case at $t_{n_2}$. Thus the temporal scope of this operation is $\interval{t_n}{t_{n_2}}$. The only interval looked up or assigned to in the rule is $\interval{t_n}{t_{n_2}}$, so the rule operates strictly within the temporal scope of the operation.

   The spatial scope for this operation is the name, the feature, and its original name. The only feature looked up or assigned to is $\lookup{\features}{\var{featureID}}$, and the only names looked up or assigned to are $\var{oldName}$ and $\var{name}$. The $\groups{}$ map is not modified or looked up in by the rule. Clearly, the rule stays within the spatial scope.
\\

\begin{lemma}
   The \rulefont{Change-Feature-Name} rule operates strictly within the scope of the \textbf{changeFeatureName} operation.
   \label{lemma:change-feature-name-scope}
\end{lemma}

\paragraph{Preserving well-formedness}
The rule does not modify any feature's existence set or type, so \wf{1\textendash 2} holds. Since it does change a name, we must look at that modification to make sure that \wf{3} is true for the altered model. A requirement for \wf{3} is that a feature has \emph{exactly} one name. 
The feature is altered thus:
\[
   \begin{split}
        (\lookup{\left(\lookup{\lookup{\names{}}{\var{name}}}{\interval{t_n}{t_{n_2}}} \assign \var{featureID}\right)}{\var{oldName}} \assign \\
        \var{clampInterval}(\lookup{\names}{\var{oldName}}, t_n), \\
   \end{split}
\]
This ensures that the feature's original name stops at $t_n$ and the new one lasts for the duration of the temporal scope $\interval{t_n}{t_{n_2}}$, ensuring that the feature has \emph{exactly} one name during the temporal scope.
Moreover \wf{3} requires that the name belongs to the same feature, and no other. This is fulfilled by 
\[
   \begin{split}
        \lookup{\lookup{\big(\lookup{\names{}}{\var{oldName}} \assign 
        \var{clampInterval}\left(\lookup{\names}{\var{oldName}}, t_n\right) \big)}{\var{name}}}{\interval{t_n}{t_{n_2}}} \\
        \assign \var{featureID} \\
   \end{split}
\]
Here, the interval containing $t_n$ in $\lookup{\names}{\var{oldName}}$ is clamped to end at $t_n$, and the resulting map is assigned $\var{featureID}$ at $\var{name}$ during the temporal scope, so the new name belongs to only the feature. This fulfils \wf{3}.
As no other part of the interval-based feature model is modified, $\wf{4\textendash 9}$ hold.
\\

\begin{lemma}
   The \rulefont{Change-Feature-Name} rule preserves well-formedness of the interval-based feature model. 
   \label{lemma:change-feature-name-well-formed}
\end{lemma}

\paragraph{Correctness of model modification}
The expected result of applying the rule is that $\lookup{\features'}{\var{featureID}} = \featurevar{F'}$ has the name $\var{name}$ during the temporal scope $\interval{t_n}{t_{n_2}}$. Indeed, due to the semantics of $\var{clampInterval}$ and assignment, for any time point $t_i$ such that $t_n \leq t_i < t_{n_2}$,
\[
   \lookup{F'_n}{t_i} = \set{\var{name}}
\]
Additionally, we should have $\lookup{\names'}{\var{name}} = \var{featureID}$. This is shown in the previous paragraph on well-formedness.
Since no other part of the interval-based feature model is altered, the rule performs as desired.
\\

\begin{lemma}
   The \rulefont{Change-Feature-Name} rule updates the interval-based feature model according to the semantics of the \textbf{changeFeatureName} operation.
   \label{lemma:change-feature-name-mod}
\end{lemma}


\section{Soundness of the Rule System}
In Section~\vref{sec:soundness-of-the-rules}, we have shown that each of the analysis rules operates within the scope of its associated operation. We have also shown that all of the rules preserve well-formedness for the interval-based feature model, and that they update the model correctly according to the semantics of their associated operations. We summarise this into three theorems that together show that the rule system is sound.

In Section~\vref{sec:soundness-of-the-rules} we have shown that each of the analysis rules operates within the scope of an operation. \footnote{See lemmas \ref{lemma:add-feature-scope}, \ref{lemma:add-group-scope}, \ref{lemma:remove-feature-scope}, \ref{lemma:remove-group-scope}, \ref{lemma:move-feature-scope}, \ref{lemma:move-group-scope}, \ref{lemma:change-feature-variation-type-scope}, \ref{lemma:change-group-variation-type-scope}, \ref{lemma:change-feature-name-scope}}
\\
\begin{theorem}
  The rule system supports local modification and verification of interval-based feature models.
\end{theorem}

Section~\vref{sec:soundness-of-the-rules} shows that each of the analysis rules preserves well-formedness of the interval-based feature model.\footnote{See lemmas \ref{lemma:add-feature-well-formed}, \ref{lemma:add-group-well-formed}, \ref{lemma:remove-feature-well-formed}, \ref{lemma:remove-group-well-formed}, \ref{lemma:move-feature-well-formed}, \ref{lemma:move-group-well-formed}, \ref{lemma:change-feature-variation-type-well-formed}, \ref{lemma:change-group-variation-type-well-formed}, \ref{lemma:change-feature-name-well-formed}}
\\
\begin{theorem}
  The rule system for local modification of interval-based feature models is well-formed.
\end{theorem}

Section~\vref{sec:soundness-of-the-rules} shows that each of the analysis rules behaves according to the semantics of the operation in question.\footnote{See lemmas \ref{lemma:add-feature-mod}, \ref{lemma:add-group-mod}, \ref{lemma:remove-feature-mod}, \ref{lemma:remove-group-mod}, \ref{lemma:move-feature-mod}, \ref{lemma:move-group-mod}, \ref{lemma:change-feature-variation-type-mod}, \ref{lemma:change-group-variation-type-mod}, \ref{lemma:change-feature-name-mod}}
\\
\begin{theorem}
  The rule system for local modification of interval-based feature models modifies the model correctly.
\end{theorem}

