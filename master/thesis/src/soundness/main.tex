\chapter{Soundness}
\label{cha:soundness}
\todo{and completeness? The operations in themselves are not complete, in the sense that there is not a one-to-one correspondence between the operations defined on temporal feature models (temporal operations) and the operations defined on feature models (model operations). However, the temporal operations are more restricted than the model operations, and any valid change within that subset to an evolution plan would be accepted by the rules. Should I try to prove completeness as well?}

\todo{Root feature requirements}

\section{Soundness for temporal feature models}
\label{sec:soundness-for-temporal-feature-models}
The temporal feature model can be viewed as a sequence of feature models associated with time points. A feature model has strict structural requirements, and the definition of a paradox is a feature model that violates these requirements. In this context, soundness means that if a rule accepts a modification, realising the modified plan results in a sequence of feature models where each is structurally sound. The soundness analysis in this chapter assumes that the original plan is sound; i.e. all resulting feature models fulfil the structural requirements. 

According to \cite{art:consistency-preserving-evolution-planning}, the structural requirements (well-formedness rules) are 
\begin{enumerate}[\itbf{WF\arabic*}, itemsep=0mm]
   \item A feature model has exactly one root feature.
   \item The root feature must be mandatory.
   \item Each feature has exactly one unique name, variation type and (potentially empty) collection of subgroups.
   \item Features are organized in groups that have exactly one variation type.
   \item Each feature, except for the root feature, must be part of exactly one group.
   \item Each group must have exactly one parent feature.
   \item Groups with types \xortype{} or \ortype{} must not contain \mandatory{} features.
   \item Groups with types \xortype{} or \ortype{} must contain at least two child features.
\end{enumerate}

One of these requirements is not upheld in this thesis, namely \itbf{WF8}. This is because it would cause unnecessary clutter in the rules, although it could be verified quite easily in an actual implementation.

\todo{This may not be a good idea, since the reasoning for scope blends with the reasoning for the other checks. Maybe divide into validate and apply, like in the implementation?}
The proof for soundness includes two parts: 
\begin{enumerate}[(i)]
   \item proving that the temporal and spatial scopes are correct, and
   \item that the checks within those scopes are sufficient for soundness.
\end{enumerate}
For each rule, the proofs are divided into these two parts.

   

\subsection{Soundness for the Add feature rule}
\label{sub:soundness-for-the-add-feature-rule}

See figure \vref{rule:add-feature} for the \rulefont{Add-Feature} rule.
Let 
\[ \textbf{addFeature}(\var{fid}, \var{name}, \var{type}, \var{parentGroupID})\text{ at }\interval{t_n}{t_m}
\shove (\names{}, \features{}, \groups{})
\]
be the initial state. Recall that this operation adds the feature with ID \var{fid} to the temporal feature model $(\names{}, \features{}, \groups{})$ from $t_n$ to $t_n$.

We argue that the temporal scope is $\interval{t_n}{t_m}$, since this is the only interval in which the temporal feature model is affected by the change. In other words, if we look at the temporal feature model as a sequence of feature models, the only models that may break as a result of this modifications, are the ones associated with time points between $t_n$ and (but not including) $t_m$. 

The well-formedness rules must hold for each feature model in the sequence; that is, for any time point $t_k$ within the scope, the feature model at $t_k$ must uphold all the rules. 

Since we give a parent group in the operation, the new feature cannot be a root feature, so \itbf{WF1} is not violated. It is also obvious that the root feature does not change its type in this rule, so \itbf{WF2} is upheld. 

However, \itbf{WF3} must be looked at more closely for this rule. Since we check that no other feature has the name \var{name} during the temporal scope, the name is unique for each affected time point. The variation type is given in the operation, and a newly added feature does not have child groups. Thus, \itbf{WF3} is upheld. Since the rule does not change the type of a group, \itbf{WF4} is not affected. As for \itbf{WF5}, since the ID of the parent group (\var{parentGroupID}) is also specified in the operation, the feature is part of exactly one group. \todo{WF6!!!}

