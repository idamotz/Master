\chapter{Soundness}
\label{cha:soundness}
% \todo{A biiiggg issue with these proofs is that I don't differentiate between the feature model before $(\names, \features, \groups)$ and modified $(\names', \features', \groups')$.}

% \todo{and completeness? The operations in themselves are not complete, in the sense that there is not a one-to-one correspondence between the operations defined on interval-based feature models (temporal operations) and the operations defined on feature models (model operations). However, the temporal operations are more restricted than the model operations, and any valid change within that subset to an evolution plan would be accepted by the rules. Should I try to prove completeness as well?}

In this chapter, we prove soundness for the analysis rules.


\todo{Elaborate the proofs kept in this chapter}
\todo{Prepare the reader for what is to come. To not overwhelm the reader with proofs, the details can be found in the appendix. Here I will emphasize how I prove things, give concrete samples, and point to where the rest of it can be found.}

We use an inductive proof structure to prove soundness and modularity for the rule system (Section~\vref{cha:a-rule-system-for-analysis-of-plan-change}). The base case is a proof by construction, in which we define a sound interval-based feature model. Using the induction hypothesis that the initial model is sound, we prove that each rule preserves soundness and operates within the previously defined scope (Section~\vref{sec:scope}). We present only two of the proofs in this chapter, as the others are much the same. The rest of the proofs can be found in Appendix~\vref{cha:remaining-soundness-proofs}.

\section{Soundness for interval-based feature models}
\label{sec:soundness-for-interval-based-feature-models}
% \todo{Argue that the rules are \emph{enough} to preserve soundness.}
% \todo{Show what needs to be proved for an initial model. ``correct by construction". In a well-formed structure. Lemma: What is a sound structure. Soundness theorem: Preserves soundness, using the lemma.}
% \todo{Proof for each rule should be lemma, and the whole system is a theorem or something:). Remember to number the lemmas}
% \todo{The structure of the proof should be introduced first}
% \todo{In the theorem, emphasize the structure of the proof (inductive)}
% \todo{Between the proofs, add some text; how to read the proof, how to understand etc. Also give context (how the lemmas are used later)}
The interval-based feature model can be viewed as a sequence of feature models associated with time points. A feature model has strict structural requirements, and the definition of a paradox is a feature model that violates these requirements. In this context, soundness means that if a rule accepts a modification, realising the modified plan results in a sequence of feature models where each is structurally sound. The soundness analysis in this chapter assumes that the original plan is sound; i.e. all resulting feature models fulfil the structural requirements. 

We must first define what it means for an interval-based feature model to be sound. Essentially, it means that if we converted the interval-based feature model into a sequence of time points associated with feature models, each feature model would be sound.


The well-formedness requirements listed in Section \vref{sub:feature-models} can be translated into rules for interval-based feature models $(\names, \features, \groups)$. We assume that the first time point in the plan is $t_0$.

\begin{enumerate}[\wf{\arabic*}, itemsep=0mm]
   \item An interval-based feature model has exactly one root feature. We assume that the root ID is $\var{rootID}$, and that $\lookup{\features}{\var{rootID}} = \featurevar{R}$. This also means that $R_e=\set{\interval{t_0}{\forever}}$ \textemdash{} the root always exists, and that $R_p = \emptyset$ \textemdash{} the root never has a parent group.
\item The root feature must be mandatory. This means that $$R_t  = \set{\intervalmapping{t_0}{\forever}{\mandatory}}$$ where $R_t$ is the types map of the root feature. 
\item At any time $t_n \geq t_0$, each feature has exactly one unique name, variation type and (potentially empty) collection of subgroups. Given a feature ID $\var{featureID}$, this means that if $\lookup{\features}{\var{featureID}} = \feature$ and $t_n \inn F_e$ , then
   \begin{enumerate}[(i)]
      \item $\lookup{F_n}{t_n} = \set{\var{name}}$ \textemdash{} the feature has exactly one name,
      \item $\lookup{\lookup{\names}{\var{name}}}{t_n} = \set{\var{featureID}}$ \textemdash{} the name is unique \emph{at the time point $t_n$},
      \item $\lookup{F_t}{t_n} = \set{\var{type}} \text{ with } \var{type} \in \set{\mandatory, \optional}$ \textemdash{} the feature has exactly one type, and
      \item $\lookup{F_c}{t_n} = C$, such that $\bigcup C$ is a set of the group IDs, and if $\var{groupID} \in \bigcup C$ and $\lookup{\groups}{\var{groupID}} = \group{}$, then $\lookup{G_p}{t_n} = \set{\var{featureID}}$ \textemdash{} if a group is listed as a subgroup of a feature, then the feature is listed as the parent of the group at the same time.
   \end{enumerate}
   \item At any time $t_n \geq t_0$, each group has exactly one variation type. Given a group ID $\var{groupID}$, this means that if $\lookup{\groups}{\var{groupID}} = \group$ and $t_n \inn G_e$, then $\lookup{G_t}{t_n} = \set{\var{type}}$ for $\var{type} \in \set{\andtype, \ortype, \xortype}$.
   \item At any time $t_n \geq t_0$, each feature, except for the root feature, must be part of exactly one group. Formally, given a feature ID $\var{featureID} \neq \var{rootID}$, if $\lookup{\features}{\var{featureID}} = \feature$, and $t_n \inn F_e$, then $\lookup{F_p}{t_n} = \set{\var{groupID}}$ with $\lookup{\groups}{\var{groupID}} = \group$, $t_n \inn G_e$, and $\var{featureID} \in \bigcup \lookup{G_c}{t_n}$. Conversely, if $\var{featureID} \in \bigcup \lookup{G_c}{t_n}$, then $\lookup{F_p}{t_n} = \var{groupID}$.
   \item At any time $t_n \geq t_0$, each group must have exactly one parent feature. Formally, given a group ID $\var{groupID}$, if $\lookup{\groups}{\var{groupID}} = \group$ and $t_n \inn G_e$, then $\lookup{G_p}{t_n} = \set{\var{featureID}}$, and $\lookup{\features}{\var{featureID}} = \feature$ with $\var{groupID} \in \bigcup \lookup{F_c}{t_n}$.
   \item At any time $t_n$, a group with types \xortype{} or \ortype{} must not contain \mandatory{} features. Formally, given a group ID $\var{groupID}$ with $\lookup{\groups}{\var{groupID}} = \group$, if $\lookup{F_t}{t_n} = \set{\var{type}}$ with $\var{type} \in \set{\xortype{}, \ortype{}}$, and if $\var{featureID} \in \bigcup \lookup{F_c}{t_n}$ and $\lookup{\features}{\var{featureID}} = \feature$, then $\lookup{F_t}{t_n} = \set{\optional}$.
\end{enumerate}

We must add two additional requirements:
\begin{enumerate}[itemsep=0mm]
   \item[\wf{8}] For a feature with ID $\var{featureID}$ such that $\lookup{\features}{\var{featureID}} = \feature$, if $t_n \notinn F_e$, then $\lookup{F_n}{t_n} = \lookup{F_t}{t_n} = \lookup{F_p}{t_n} = \lookup{F_c}{t_n} = \emptyset$, and for all keys $\var{name}$ in $\names$, $\var{featureID} \notin \lookup{\lookup{\names}{\var{name}}}{t_n}$ \textemdash{} no name belongs to the feature. Similarly, for a group with ID $\var{groupID}$ such that $\lookup{\groups}{\var{groupID}} = \group$, if $t_n \notinn G_e$, then $\lookup{G_t}{t_n} = \lookup{G_p}{t_n} = \lookup{G_c}{t_n} = \emptyset$. In other words, a feature or a group which does not exist cannot have a name, a type, a parent, or a child.
   \item[\wf{9}] The interval-based feature model contains no cycles, which means that at any time point $t_n \geq t_0$, for any feature or group that exists at $t_n$, if we follow the parent chain upwards, we never encounter the same feature or group twice. In other words, no feature or group is its own ancestor.
\end{enumerate}

Together, these requirements form the basis of the soundness proofs. We assume that the original plan is sound, so each of these requirements is assumed to be true for the original interval-based feature model. Furthermore, we prove that the requirements must still hold for the updated model if the rule can be applied.

\section{Soundness of each rule}
\label{sec:soundness-of-each-rule}
In the following sections, we prove that each rule is sound, and conclude that the system is sound.

For each rule, the proof for soundness includes three parts: 
\begin{enumerate}[(i)]
   \item proving that the rule operates strictly within the previously defined temporal and spatial scopes (Section~\vref{sec:define-the-scope}), 
   \item that the rule preserves well-formedness, as defined in the above requirements \wf{1\textendash 9}, and
   \item that the rule updates the model correctly, preserving soundness as well as respecting the semantics of the operation.
\end{enumerate}



\section{Soundness of the Add feature rule}
\label{sec:soundness-of-the-add-feature-rule}

\todo{Use a proof environment that clearly separate definitions from proofs. The proofs should contain syntactic correctness, not semantic (that part is moved out of the proof)}

See figure \vref{rule:add-feature} for the \rulefont{Add-Feature} rule.
Let 
\begin{equation*}
   \begin{split}
      \textbf{addFeature}(\var{featureID}, \var{name}, \var{type}, \var{parentGroupID})\text{ at }\interval{t_n}{t_m}
      \shove \\
   (\names{}, \features{}, \groups{})
\end{split}
\end{equation*}
be the initial state. Recall that this operation adds the feature with ID \var{featureID} to the temporal feature model $(\names{}, \features{}, \groups{})$ from $t_n$ to $t_m$. We assume that $(\names{}, \features{}, \groups{})$ is well-formed, as defined in \itbf{TFMWF1-9}.

\todo{Show that the feature model outside the scope is unchanged}

\paragraph{Scope}

Recall from section \vref{sec:define-the-scope} that the temporal scope of this operation is $\interval{t_n}{t_m}$, and the spatial scope is the feature itself, the parent group and the name. 

In the rule, we look up only the feature ID, the parent group ID, and the name, and update only the name, feature, and parent group. Thus, the rule operates within the spatial scope of the operation. Furthermore, the only interval looked up in the interval maps and sets of the model is $\interval{t_n}{t_m}$, which is exactly the temporal scope of the rule. Hence the rule operates strictly within the temporal and spatial scopes of the operation.
\\

\begin{lemma}
   The \rulefont{Add-Feature} rule operates strictly within the scope of the \textbf{addFeature} operation.
   \label{lemma:add-feature-scope}
\end{lemma}


\paragraph{Preserving well-formedness}

If the rule is applied, the well-formedness requirements must hold for the updated feature model. 

Since the rule checks that the feature does not already exist during the temporal scope, it is impossible that $\var{featureID} = \var{rootID}$. Thus the rule does not affect the root feature, and \itbf{TFMWF1} and \itbf{TFMWF2} hold for the updated temporal feature model.

Because we assume that \itbf{TFMWF8} holds for the original model, and the feature does not exist during $\interval{t_n}{t_m}$, the feature has no name, type, or subgroups in the original plan. When we add the feature to the feature model using \var{setFeatureAttributes}, we give the feature exactly one name and one type during the temporal scope, and the set of child groups is empty. The temporal scope is also added to the feature's existence set, so only the new feature has the ID $\var{featureID}$ during the temporal scope. To link the feature ID to the name, the rule sets the feature ID as the value at key $\var{name}$ in the $\names{}$ map during the temporal scope, so the name is unique during the temporal scope.
Consequently, \itbf{TFMWF3} holds.

The rule does not modify the parent group's variation type, so \itbf{TFMWF4} is preserved in the modified temporal feature model.

Similarly to the argument for \itbf{TFMWF3}, the parent group ID is uniquely defined for the feature in \var{setFeatureAttributes}, and \var{featureID} is added to the parent group's set of child features, so the new feature is part of exactly one group. Since we do not remove any other feature IDs from the parent group's set of features, and as we already established that the new feature is not the root feature, \itbf{TFMWF5} is preserved.

The new feature does not have any subgroups during the temporal scope, and we do not modify the parent group's parent feature. Under the assumption that \itbf{TFMWF6} holds in the original model, it still holds after applying the \rulefont{Add-Feature} rule.

The rule verifies that all of the parent group's types are compatible with the added feature's type during the temporal scope, so \itbf{TFMWF7} holds after applying the rule.

Since the rule adds the temporal scope to the new feature's existence table, and since the parent group exists in the original plan, \itbf{TFMWF8} is preserved after the rule is applied.

It is furthermore impossible that adding this feature creates a cycle in the modified model. The new feature has no subgroups, so it cannot be part of a cycle. Because of the assumption that \itbf{TFMWF9} holds in the original plan, and applying the rule does not introduce a cycle, this requirement still holds. 

As the rule operates within the scope (lemma \vref{lemma:add-feature-scope}), it does not affect any other part of the plan.

We conclude that the \rulefont{Add-Feature} rule preserves well-formedness for the temporal feature model, according to well-formedness rules \itbf{TFMWF1-9}. 
\\
\begin{lemma}
   The \rulefont{{Add-Feature}} rule preserves well-formedness of the temporal feature model.
   \label{lemma:add-feature-well-formed}
\end{lemma}

\todo{Mention that the scope part ensures that no other part of the plan is affected}
\paragraph{Correctness of model modification}

The operation is intended to add the feature with ID \var{featureID} to the temporal feature model during the interval $\interval{t_n}{t_m}$. 

After adding the feature to the temporal feature model, looking up the name $\var{name}$ in the $\names{}$ map at interval key $\interval{t_n}{t_m}$ should give the value $\var{featureID}$. Indeed, since the $\names{}$ map is updated thus:
\[
   \lookup{\lookup{\names}{\var{name}}}{\interval{t_n}{t_m}} \assign \var{featureID}
\]
, then due to the semantics of map assignment (definition \vref{def:map}), and lookup in interval maps (definition \vref{def:interval-map}),
\[
   \lookup{\lookup{\names}{\var{name}}}{\interval{t_n}{t_m}} = \set{\var{featureID}}
\]
will hold.

Similarly, if we wish to lookup information about the feature during the interval $\interval{t_n}{t_m}$ in the modified model, the results should match the information in the operation. The rule assigns 
\begin{equation*}
   \begin{split}
   \var{setFeatureAttributes}(\lookup{\features{}}{\var{featureID}}, \interval{t_n}{t_m}, \var{name}, \var{type}, \\
   \var{parentGroupID})
   \end{split}
\end{equation*}

to $\lookup{\features}{\var{featureID}}$. 

According to the semantics of assignment (section \vref{sub:maps}) and $\var{setFeatureAttributes}$ (figure \vref{fun:set-feature-attributes}), and given that $\lookup{\features}{\var{featureID}} = \feature$, then 
\begin{align}
   \interval{t_n}{t_m} &\in F_e && \text{the feature exists} \tag{1} \label{add-feature1}\\
   \lookup{F_n}{\interval{t_n}{t_m}} &= \set{\var{name}} && \text{the feature has the expected name} \tag{2} \label{add-feature2}\\
   \lookup{F_t}{\interval{t_n}{t_m}} &= \set{\var{type}}  && \text{the feature has the expected type} \tag{3} \label{add-feature3}\\
   \lookup{F_p}{\interval{t_n}{t_m}} &= \set{\var{parentGroupID}}  && \text{the feature has the expected parent group} \tag{4} \label{add-feature4}\\
   \lookup{F_c}{\interval{t_n}{t_m}} &= \emptyset && \text{the feature has no children} \tag{5} \label{add-feature5}
\end{align}

Statement (\ref{add-feature1}) holds due to the line $F_e \cup \interval{t_n}{t_m}$ in $\var{setFeatureAttributes}$. The next four hold due to both premises in the rule and modifications in the function. Due to the premise $\interval{t_n}{t_m} \not \innr F_e$, which means that the feature does not previously exist at any point during the interval, and since \itbf{TFMWF8} is assumed to hold for the original model, the original feature does not have a name, type, parent group or child groups during the interval. In the function $\var{setFeatureAttributes}$, the name is added ($\lookup{F_n}{\interval{t_{start}}{t_{end}}} \assign \var{name}$), and so is the type ($\lookup{F_t}{\interval{t_{start}}{t_{end}}} \assign \var{type}$) and the parent group ($\lookup{F_t}{\interval{t_{start}}{t_{end}}} \assign \var{parentGroupID}$). The child groups are not modified, and so (\ref{add-feature5}) holds.

The child features of the group must also be updated according to the semantics of the operation. After applying the rule, given that $\lookup{\groups}{\var{parentGroupID}} = \group{}$, 
\[
   \var{featureID} \in \bigcup \lookup{G_c}{\interval{t_n}{t_m}}
\]
, meaning that the feature is in the parent group's set of child features in the updated model. This holds because $\lookup{\groups}{\var{parentGroupID}}$ is assigned $\var{addChildFeature}(\lookup{\groups}{\var{parentGroupID}}, \interval{t_n}{t_m}, \var{featureID})$ (see figure \vref{fun:add-child-feature}), which modifies $G_c$ by adding $\var{featureID}$ to the set of child features at interval key $\interval{t_n}{t_m}$.
\\

\begin{lemma}
   The \rulefont{Add-Feature} rule updates the temporal feature model according to the semantics of the \textbf{addFeature} operation.
   \label{lemma:add-feature-mod}
\end{lemma}



\subsection{Soundness of the Move feature rule}
\label{sub:soundness-of-the-move-reature-rule}

See figure \vref{rule:move-feature} for the \rulefont{Move-Feature} rule. Let 
\begin{equation*}
   \begin{split}
      \textbf{moveFeature}(\var{featureID}, \var{newParentID}) \text{ at } t_n \shove \\
      (\names, \features, \groups)
   \end{split}
\end{equation*}
be the initial state, and
\[
   (\names', \features', \groups')
\]
be the result state after applying the \rulefont{Move-Feature} rule. Recall that this operation moves the feature with ID $\var{featureID}$ to the group with ID $\var{newParentID}$.

\paragraph{Scope}
\label{par:scope}
Recall that the temporal scope of the move-feature rule is $\interval{t_n}{t_k}$ (section \vref{sec:define-the-scope}), where $t_k$ is the time point at which the feature is originally planned to be moved or is removed. In the rule, this scope is identified by 
   \[
      \containing{F_p}{t_n} = \set{\interval{t_{p_1}}{t_{p_2}}}
   \]
   Here, the time point $t_n$ for moving the feature is looked up in the feature's parent map's set of interval keys, and the expected result is $\set{\interval{t_{p_1}}{t_{p_2}}}$. This means that there is a mapping $\intervalmapping{t_{p_1}}{t_{p_2}}{\var{parentGroupID}}$ in $F_p$, with $\var{parentGroupID}$ being the ID of the feature's parent group at time $t_n$, and this group stops being the feature's parent at $t_{p_2}$. Thus the temporal scope of this operation is $\interval{t_n}{t_{p_2}}$. The only interval looked up or assigned to in the rule is $\interval{t_n}{t_{p_2}}$, but it is necessary to also look at the cycle detection algorithm in section \vref{sub:move-algorithm}, since this is also referenced in the rule by $\neg \var{createsCycle}$. Here, $t_{p_2}$ is called $t_e$, and the algorithm states that it only looks at time points between $t_n$ and $t_e$. Thus the rule operates strictly within the temporal scope of the \textbf{moveFeature} operation.

   The spatial scope for this operation is defined as the \emph{ancestors which the feature and the target group do not have in common}. In other words, the \emph{new} ancestors of the feature after applying the rule. In the rule itself, only the feature with ID $\var{featureID}$ and its new parent group with ID $\var{newParentID}$ are looked up. However, the cycle detection algorithm must also be considered. Here, the ancestors of both the feature and the group at $t_n$ are looked up, the first ancestor they have in common identified, and the new ancestors are collected into a list. If one of them is moved before $t_e$, the list is updated. Hence the algorithm's spatial scope is indeed the feature's new ancestors and the feature itself, and so the rule operates within the defined spatial scope.
\\

\begin{lemma}
   The \rulefont{Move-Feature} rule operates strictly within the scope of the \textbf{moveFeature} operation.
   \label{lemma:move-feature-scope}
\end{lemma}

\paragraph{Preserving well-formedness}
Since the rule verifies that the feature has a parent group, the feature being moved is not the root. Thus \wf{1} and \wf{2} hold. The rule does not update the name, type or subgroups of the feature, so \wf{3} is true for the updated model. Nor does it modify the target group's type or parent feature, so \wf{4}, \wf{6}, and \wf{7} also hold.

The modification made to $\lookup{\features}{\var{featureID}}$ is to the parent group map $F_p$ by 
\[
   F_p' = \lookup{\var{clampInterval}(F_p, \, t_n)}{\interval{t_n}{t_{p_2}}} \assign \var{newParentID}
\]
As discussed in earlier sections (e.g. \vref{sub:soundness-of-the-remove-feature-rule}), $\var{clampInterval}$ replaces a mapping $\intervalmapping{t_i}{t_j}{v}$ by $\intervalmapping{t_i}{t_n}{v}$, with $t_n \leq t_j$. Thus $\var{clampInterval}(F_p, \, t_n)$ has no parent group during the temporal scope. The subsequent assignment of $\var{newParentID}$ to $\interval{t_n}{t_{p_2}}$ ensures that the feature has exactly one parent group during the temporal scope. This relation is reflected in the $\groups{}'$ map, with
\begin{align*}
   \groups' &= \\
   \lookup{&\big(\lookup{\groups{}}{\var{oldParentID}} \\
   \assign & \var{removeFeatureAt}\left(\lookup{\groups{}}{\var{oldParentID}}, \var{featureID}, t_n\right)\big)}{\var{newParentID}} \\
   \assign& \var{addChildFeature}(\lookup{\groups{}}{\var{newParentID}}, \interval{t_n}{t_{p_2}}, \var{featureID})
\end{align*}
The feature is added to the target group by $\var{addChildFeature}$ during the interval $\interval{t_n}{t_{p_2}}$, and removed from the original parent group by $\var{removeFeatureAt}$. Consequently \wf{5} holds for the updated model. 

As the rule does not alter the feature's existence set, \wf{8} is preserved. 

The intention of the cycle detection algorithm in section \vref{sub:move-algorithm} is to uphold \wf{9}. Given the assumption that the original temporal feature model contains no cycles, if the altered model contains a cycle then the $\textbf{moveFeature}$ operation introduced it, and the feature being moved must be part of the cycle. This could only happen if the feature became part of its own subtree during the temporal scope, which means that at some point, the feature occurs in its own list of ancestors. The algorithm looks at the feature's \emph{new} ancestors, meaning the ancestors that the feature does not have in the original plan, but does in the new one. It then checks that none of those ancestors are moved to the feature's subtree. Thus the rule preserves \wf 9.
\\

\begin{lemma}
   The \rulefont{Move-Feature} rule preserves well-formedness of the temporal feature model. 
   \label{lemma:move-feature-well-formed}
\end{lemma}

\paragraph{Correctness of model modification}
The operation is intended to move the feature with ID $\var{featureID}$ to the group with ID $\var{newParentID}$ during the temporal scope $\interval{t_n}{t_{p_2}}$. After applying the \rulefont{Move-Feature} rule, the only differences between the original and modified temporal feature model should be
\begin{enumerate}[(i)]
   \item The feature's parent group should be $\var{newParentID}$ during the temporal scope \label{move-feature-parentgroup}
   \item The feature should not appear in the original parent group's set of subfeatures during the temporal scope \label{move-feature-oldparent}
   \item The feature should appear in the new parent group's set of subfeatures \label{move-feature-newparent}
\end{enumerate}

Given the modified map of parent groups $F'_p$ and the original map $F_p$, we have that
\[
   F'_p = \lookup{\var{clampInterval}(F_p, t_n)}{\interval{t_n}{t_{p_2}}} \assign \var{newParentID}
\]
This statement assigns $\var{newParentID}$ to the temporal scope $\interval{t_n}{t_{p_2}}$ after applying $\var{clampInterval}(F_p, t_n)$, meaning that the original parent mapping is shortened to end at $t_n$, and a new mapping $\intervalmapping{t_n}{t_{p_2}}{\var{newParentID}}$ is inserted. By semantics of assignment, it is clear that $\lookup{F'_p}{\interval{t_n}{t_{p_2}}} = \set{\var{newParentID}}$, which is the desired result and fulfils \ref{move-feature-parentgroup}.



\begin{lemma}
   The \rulefont{Move-Feature} rule updates the temporal feature model according to the semantics of the \textbf{moveFeature} operation.
   \label{lemma:move-feature-mod}
\end{lemma}


\section{Soundness of the Rule System}
In Section~\vref{sec:soundness-of-the-rules}, we have shown that each of the analysis rules operates within the scope of its associated operation. We have also shown that all of the rules preserve well-formedness for the interval-based feature model, and that they update the model correctly according to the semantics of their associated operations. We summarise this into three theorems that together show that the rule system is sound.

In Section~\vref{sec:soundness-of-the-rules} we have shown that each of the analysis rules operates within the scope of an operation. \footnote{See lemmas \ref{lemma:add-feature-scope}, \ref{lemma:add-group-scope}, \ref{lemma:remove-feature-scope}, \ref{lemma:remove-group-scope}, \ref{lemma:move-feature-scope}, \ref{lemma:move-group-scope}, \ref{lemma:change-feature-variation-type-scope}, \ref{lemma:change-group-variation-type-scope}, \ref{lemma:change-feature-name-scope}}
\\
\begin{theorem}
  The rule system supports local modification and verification of interval-based feature models.
\end{theorem}

Section~\vref{sec:soundness-of-the-rules} shows that each of the analysis rules preserves well-formedness of the interval-based feature model.\footnote{See lemmas \ref{lemma:add-feature-well-formed}, \ref{lemma:add-group-well-formed}, \ref{lemma:remove-feature-well-formed}, \ref{lemma:remove-group-well-formed}, \ref{lemma:move-feature-well-formed}, \ref{lemma:move-group-well-formed}, \ref{lemma:change-feature-variation-type-well-formed}, \ref{lemma:change-group-variation-type-well-formed}, \ref{lemma:change-feature-name-well-formed}}
\\
\begin{theorem}
  The rule system for local modification of interval-based feature models is well-formed.
\end{theorem}

Section~\vref{sec:soundness-of-the-rules} shows that each of the analysis rules behaves according to the semantics of the operation in question.\footnote{See lemmas \ref{lemma:add-feature-mod}, \ref{lemma:add-group-mod}, \ref{lemma:remove-feature-mod}, \ref{lemma:remove-group-mod}, \ref{lemma:move-feature-mod}, \ref{lemma:move-group-mod}, \ref{lemma:change-feature-variation-type-mod}, \ref{lemma:change-group-variation-type-mod}, \ref{lemma:change-feature-name-mod}}
\\
\begin{theorem}
  The rule system for local modification of interval-based feature models modifies the model correctly.
\end{theorem}

