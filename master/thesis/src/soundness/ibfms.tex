\section{Soundness for interval-based feature models}
\label{sec:soundness-for-interval-based-feature-models}
% \todo{Argue that the rules are \emph{enough} to preserve soundness.}
% \todo{Show what needs to be proved for an initial model. ``correct by construction". In a well-formed structure. Lemma: What is a sound structure. Soundness theorem: Preserves soundness, using the lemma.}
% \todo{In the theorem, emphasize the structure of the proof (inductive)}
% \todo{Between the proofs, add some text; how to read the proof, how to understand etc. Also give context (how the lemmas are used later)}
The interval-based feature model can be viewed as a sequence of feature models associated with time points. A feature model has strict structural requirements, and the definition of a paradox is a feature model that violates these requirements. In this context, soundness means that if a rule accepts a modification, realising the modified plan results in a sequence of feature models where each is structurally sound. The soundness analysis in this chapter assumes that the original plan is sound; i.e. all resulting feature models fulfil the structural requirements. 

We must first define what it means for an interval-based feature model to be sound. Essentially, it means that if we converted the interval-based feature model into a sequence of time points associated with feature models, each feature model would be sound.


The well-formedness requirements listed in Section \vref{sub:feature-models} can be translated into rules for interval-based feature models $(\names,\allowbreak \features, \allowbreak \groups)$. We assume that the first time point in the plan is $t_0$.

\begin{enumerate}[\wf{\arabic*}, itemsep=0mm]
   \item An interval-based feature model has exactly one root feature. We assume that the root ID is $\var{rootID}$, and that $\lookup{\features}{\var{rootID}} = \featurevar{R}$. This also means that $R_e=\set{\interval{t_0}{\forever}}$ \textemdash{} the root always exists, and that $R_p = \emptyset$ \textemdash{} the root never has a parent group.
\item The root feature must be mandatory. This means that $$R_t  = \set{\intervalmapping{t_0}{\forever}{\mandatory}}$$ where $R_t$ is the types map of the root feature. 
\item At any time $t_n \geq t_0$, each feature has exactly one unique name, variation type and (potentially empty) collection of subgroups. Given a feature ID $\var{featureID}$, this means that if $\lookup{\features}{\var{featureID}} = \feature$ and $t_n \inn F_e$ , then
   \begin{enumerate}[(i)]
      \item $\lookup{F_n}{t_n} = \set{\var{name}}$ \textemdash{} the feature has exactly one name,
      \item $\lookup{\lookup{\names}{\var{name}}}{t_n} = \set{\var{featureID}}$ \textemdash{} the name is unique \emph{at the time point $t_n$},
      \item $\lookup{F_t}{t_n} = \set{\var{type}} \text{ with } \var{type} \in \set{\mandatory, \optional}$ \textemdash{} the feature has exactly one type, and
      \item $\lookup{F_c}{t_n} = C$, such that $\bigcup C$ is a set of the group IDs, and if $\var{groupID} \in \bigcup C$ and $\lookup{\groups}{\var{groupID}} = \group{}$, then $\lookup{G_p}{t_n} = \set{\var{featureID}}$ \textemdash{} if a group is listed as a subgroup of a feature, then the feature is listed as the parent of the group at the same time.
   \end{enumerate}
   \item At any time $t_n \geq t_0$, each group has exactly one variation type. Given a group ID $\var{groupID}$, this means that if $\lookup{\groups}{\var{groupID}} = \group$ and $t_n \inn G_e$, then $\lookup{G_t}{t_n} = \set{\var{type}}$ for $\var{type} \in \set{\andtype, \ortype, \xortype}$.
   \item At any time $t_n \geq t_0$, each feature, except for the root feature, must be part of exactly one group. Formally, given a feature ID $\var{featureID} \neq \var{rootID}$, if $\lookup{\features}{\var{featureID}} = \feature$, and $t_n \inn F_e$, then $\lookup{F_p}{t_n} = \set{\var{groupID}}$ with $\lookup{\groups}{\var{groupID}} = \group$, $t_n \inn G_e$, and $\var{featureID} \in \bigcup \lookup{G_c}{t_n}$. Conversely, if $\var{featureID} \in \bigcup \lookup{G_c}{t_n}$, then $\lookup{F_p}{t_n} = \var{groupID}$.
   \item At any time $t_n \geq t_0$, each group must have exactly one parent feature. Formally, given a group ID $\var{groupID}$, if $\lookup{\groups}{\var{groupID}} = \group$ and $t_n \inn G_e$, then $\lookup{G_p}{t_n} = \set{\var{featureID}}$, and $\lookup{\features}{\var{featureID}} = \feature$ with $\var{groupID} \in \bigcup \lookup{F_c}{t_n}$.
   \item At any time $t_n$, a group with types \xortype{} or \ortype{} must not contain \mandatory{} features. Formally, given a group ID $\var{groupID}$ with $\lookup{\groups}{\var{groupID}} = \group$, if $\lookup{F_t}{t_n} = \set{\var{type}}$ with $\var{type} \in \set{\xortype{}, \ortype{}}$, and if $\var{featureID} \in \bigcup \lookup{F_c}{t_n}$ and $\lookup{\features}{\var{featureID}} = \feature$, then $\lookup{F_t}{t_n} = \set{\optional}$.
\end{enumerate}

We must add two additional requirements:
\begin{enumerate}[itemsep=0mm]
   \item[\wf{8}] For a feature with ID $\var{featureID}$ such that $\lookup{\features}{\var{featureID}} = \feature$, if $t_n \notinn F_e$, then $\lookup{F_n}{t_n} = \lookup{F_t}{t_n} = \lookup{F_p}{t_n} = \lookup{F_c}{t_n} = \emptyset$, and for all keys $\var{name}$ in $\names$, $\var{featureID} \notin \lookup{\lookup{\names}{\var{name}}}{t_n}$ \textemdash{} no name belongs to the feature. Similarly, for a group with ID $\var{groupID}$ such that $\lookup{\groups}{\var{groupID}} = \group$, if $t_n \notinn G_e$, then $\lookup{G_t}{t_n} = \lookup{G_p}{t_n} = \lookup{G_c}{t_n} = \emptyset$. In other words, a feature or a group which does not exist cannot have a name, a type, a parent, or a child.
   \item[\wf{9}] The interval-based feature model contains no cycles, which means that at any time point $t_n \geq t_0$, for any feature or group that exists at $t_n$, if we follow the parent chain upwards, we never encounter the same feature or group twice. In other words, no feature or group is its own ancestor.
\end{enumerate}

Together, these requirements form the basis of the soundness proofs. We assume that the original plan is sound, so each of these requirements is assumed to be true for the original interval-based feature model. Furthermore, we prove that the requirements must still hold for the updated model if the rule can be applied.

\section{Soundness of each rule}
\label{sec:soundness-of-each-rule}
In the following sections, we prove that each rule is sound, and conclude that the system is sound. We rely upon the above defined well-formedness requirements \wf{1\textendash 9} to show this.

For each rule, the proof for soundness includes three parts: 
\begin{enumerate}[(i)]
   \item proving that the rule operates strictly within the previously defined temporal and spatial scopes (Section~\vref{sec:define-the-scope}), 
   \item that the rule preserves well-formedness, as defined in the above requirements \wf{1\textendash 9}, and
   \item that the rule updates the model correctly, preserving soundness as well as respecting the semantics of the operation.
\end{enumerate}

These parts are concluded with a lemma for each rule, and the lemmas are finally used to show that the entire rule system is correct.
