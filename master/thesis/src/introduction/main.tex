\part{Introduction and Background}

\chapter{Introduction}
% Giving the thesis a purpose (not how, but why)
%   Mention problems in industry practice (mention software product lines)
%   Say what software product lines are for, not technicalities, why they exist
%   Talk about why planning is difficult, when the project size is increasing
%   -> Changing the plan is even more difficult, and there is a lack of automated tools for checking that the changes don't break anything
%   Drawing a picture of the domain and the problem
%   My thesis is addressing these challenges by looking at these research questions or goals
% Last part of introduction: research questions/goals (base structure of thesis on this) - 3 is a good number
% They are general. What, how, (never why)
% In this thesis, I will address these questions through these activities ...(will not tackle everything, so make the purpose clear)
% Start broader, chunk down to more detail
% Can take the goals up again in the following sections (mention the research questions when tackling them)
% How the sensor should read my master's thesis. How will she know what my contribution is, where it comes in. "In chapter 2 I will give the background, limited to the scope of my thesis". Make it clear that I'm not trying to cover everything, and give the logic/structure of the thesis.

Storyline: SPLs -> planning -> existing solution (LTEP) -> problem with that solution -> goal

High-level introduction to software product lines

Can talk about static analysis in terms of related work or background.
-> Semantic-based analysis
-> Forward analysis and scope (live variable analysis etc)
- Bring up research questions after this

Reference previous work clearly, talk about how my work builds upon it.

\section{Contributions}

\section{Research questions}
\label{sec:research-questions}

\todo{Make goal higher-level (objective), go into more detail in research questions}
\todo{Rephrase as questions, highlighting the problems}
\begin{enumerate}[\itbf{RG\arabic*}, itemsep=0mm]
   \item Define a representation for feature model evolution plans which enables local verification. (rephrase)
   \item Identify the scope for each operation.
   \item Create a semantics for local soundness checking of each operation.
\end{enumerate}

\todo{Describe the goals in more detail}
\todo{It should be clear to the reader that I have soundness proofs}


\chapter{Background} 

\section{Software product lines}
\subsection{Feature models}
\subsection{Feature model evolution plans}
% Software product lines, previous publication (pinpoint what can be improved), SOS rules and operational semantics, and what is modular analysis
% Background should not contain many of my results, but rather retelling what software product lines are, what operational semantics are, existing technologies
% % What is the difficulty of doing analysis on it? Just explaining the structures does not convey why it is worth spending time to solve the problems. Hint at what is coming, that this is something I will tackle in the contribution. 
% Can't be technical in the introduction, the background is where I introduce the terms and definitions and technical details. And the implications of these details.
% Explain what I mean by "modularity", explain why the previous work is not modular. 
% What do I assume, and what do I focus on
