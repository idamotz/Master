The experience from reuse projects in the 1990s shows that without proper planning the costs for reuse may be higher than for developing the artefacts from scratch. It is therefore crucial to plan beforehand the products for which reuse is sensible, together with the features that characterise these products. The planning for reuse continues throughout the whole development process. \cite{book:introduction-to-spl}

To facilitate mass customisation [...] the platform must provide the means to satisfy different stakeholder requirements. For this purpose the concept of variability is introduced in the platform. As a consequence of applying this concept, the artefacts that can differ in the applications of the product line are modelled using variability. \cite{book:introduction-to-spl}

A major strength of software product line engineering is the generation of a multitude of product variants at a reasonable cost. \cite{book:introduction-to-spl}.

\textit{Differences from single-system engineering}
Domain requirements engineering differs from requirements engineering for single systems because:
\begin{itemize}
   \item The requirements are analysed to identify those that are common to all applications and those that are specific for particular applications (i.e. that differ among several applications).
   \item The possible choices with regard to requirements are explicitly documented in the variability model, which is an abstraction of the variability of the domain requirements.
   \item Based on the input from product management, domain requirements engineering anticipates prospective changes in requirements, such as laws, standards, technology changes, and market needs for future applications.
\end{itemize}
\cite{book:introduction-to-spl}

\paragraph{Domain Variability Model}
The domain variability model defines the variability of the software product line. It defines what can vary, i.e. it introduces variation points for the product line. It also defines the types of variation offered for a particular variation point, i.e. it defines the variants offered by the product line. Moreover, the domain variability model defines variability dependencies and variability constraints which have to be considered when deriving product line applications. \cite{book:introduction-to-spl}

Feature is an end-user visible characteristic of a system \cite{book:introduction-to-spl}

Feature modeling is considered as ``the greatest contribution of domain engineering to software engineering". \cite{art:formal-semantics-and-verification-for-feature-modeling}

Conceptual relationships among features can be expressed by a feature model as proposed by Kang et al. \cite{art:formal-semantics-and-verification-for-feature-modeling}

A feature diagram provides a graphical tree-like notation that shows the hiararchical organization of features. The root of the tree represents a \textit{concept} node. All other nodes represent different types of features. \cite{art:formal-semantics-and-verification-for-feature-modeling}

[...] All program analysis should be \emph{semantics based}: this means that the information obtained from the analysis can be proved to be safe (or correct) with respect to a semantics of the programming language. [...] However, we should stress that we do \emph{not} suggest that program analyses be \emph{semantics directed}: this would mean that the structure of the program analysis should reflect the structure of the semantics and this will be the case only for a few approaches which are not covered in this book. \cite{book:principles-of-program-analysis}
