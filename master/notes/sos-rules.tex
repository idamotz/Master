\section{SOS rules}
\label{sec:sos-rules}
\todo{Consider having two rules for each operation; one for validating and one for updating. Alternatively use functions on the right-hand side of $\transition$.}

\todo{Explain how to read the rules}

\subsection{Add feature rule}
\label{sub:add-feature-rule}

\begin{figure}[h]
    \renewcommand{\arraystretch}{1.1}
    \sossize$$\begin{array}{c}
      \ntyperule{Add-Feature}
      {\\
        \interval{t_n}{t_m} \not \innr F_e \qquad
        \interval{t_n}{t_m} \inn G_e \qquad
        \names{}[\var{name}]\interval{t_n}{t_m} = \emptyset \\
        \features{}[\var{fid}] = \left( F_e,\, F_n,\, F_t,\, F_p,\, F_c \right) \\
        \groups{}[\var{parentGroupID}] = \left(G_e,\, G_t , \, G_p,\, G_c \right)\\
        \forall \var{gt}_{\in G_t\interval{t_n}{t_m}} (\var{compatibleTypes}(\var{gt}, \var{type})) \\
      }
      {
        \textbf{addFeature}(\var{fid}, \var{name}, \var{type}, \var{parentGroupID})\text{ at }\interval{t_n}{t_m} \shove \\
         (\names{}, \features{}, \groups{})\\
        \transition\\
        (\names{}[\var{name}]\interval{t_n}{t_m} \assign \var{fid},  \\
        {\features{}[\var{fid}]} \assign 
        \var{setFeatureAttributes}(\features{}[\var{fid}], \interval{t_n}{t_m}, 
        \var{name}, \var{type}, \var{parentGroupID}),\\
        {\groups{}[\var{parentGroupID]}} \assign 
        \var{addChildFeature}(\groups{}[\var{parentGroupID}], \interval{t_n}{t_m}, \var{fid}))
    }
    \end{array}$$
    \caption{\label{rule:add-feature}}
\end{figure}

Figure \vref{rule:add-feature}  describes the semantics of the \textbf{addFeature} operation. 
To add a feature during the interval $\interval{t_n}{t_m}$, its ID cannot exist exist during the interval ($\interval{t_n}{t_m} \not \innr F_e$). The parent feature must exist ($\interval{t_n}{t_m} \inn G_e$), and the types it has during the interval must be compatible with the type of the 
added feature ($\forall \var{gt}_{\in G_t\interval{t_n}{t_m}} (\var{compatibleTypes}(\var{gt}, \var{type}))$). The name of the feature must be available during the interval ($\names{}[\var{name}]\interval{t_n}{t_m} = \emptyset$). Notice that the default value in the $\features{}$ map lets us treat a failed lookup as a feature, thus allowing us to express the semantics of adding a feature using only one rule. 

To make the rule tidier, we use three helper functions: $\var{compatibleTypes}$ (Figure \vref{fun:compatible-types}), $\var{setFeatureAttributes}$ (Figure \vref{fun:set-feature-attributes}), and $\var{addChildFeature}$ (Figure \vref{fun:add-child-feature}). 

\begin{figure}[h]
  \begin{minted}{text}
compatibleTypes(AND, _) = True
compatibleTypes(_, optional) = False
compatibleTypes(_, _) = True
  \end{minted}
  \caption{\label{fun:compatible-types}}
\end{figure}

\begin{figure}[h]
  \begin{minted}[escapeinside=||]{text}
setFeatureAttributes(|$(F_e, F_n, F_t, F_p, F_c)$|, |$\interval{t_{start}}{t_{end}}$|, name, type, parentGroupID)
  = |$($| |$F_e \cup \interval{t_{start}}{t_{end}}$|
    , |$F_n$|[|$\interval{t_{start}}{t_{end}}$|] |$\assign$| name
    , |$F_t$|[|$\interval{t_{start}}{t_{end}}$|] |$\assign$| type
    , |$F_p$|[|$\interval{t_{start}}{t_{end}}$|] |$\assign$| parentGroupID
    , |$F_c$| |$)$|
     \end{minted}
  \caption{\label{fun:set-feature-attributes}}
\end{figure}

\begin{figure}[h]
  \begin{minted}[escapeinside=||]{text}
addChildFeature(|$(G_e, G_t, G_p, G_c)$|, |$\interval{t_{start}}{t_{end}}$|, fid)
  = |$\left(G_e, G_t, G_p , G_c \interval{t_{start}}{t_{end}} \addassign \var{fid}\right)$|
  \end{minted}
  \caption{\label{fun:add-child-feature}}
\end{figure}


\subsection{Add group rule}
\label{sub:add-group-rule}
The rule in figure \vref{rule:add-group} describes the conditions which must be in place to add a (pre-existing or fresh) group to the FMEP during an interval $\interval{t_n}{t_m}$. The group must not already exist in the plan during the interval $\interval{t_n}{t_m} \notinnr G_e$, and the parent feature must exist for the duration of the interval $\interval{t_n}{t_m} \inn F_e$. The group ID is added to the parent feature's map of child groups with the interval as key, and the attributes specified are added to the group entry in the $\groups$ map.

\begin{figure}[!h]
    \renewcommand{\arraystretch}{1.1}
    \sossize$$\begin{array}{c}
      \ntyperule{Add-Group}
      {\\
        \interval{t_n}{t_m} \notinnr G_e \qquad \interval{t_n}{t_m} \inn F_e \\
        \lookup{\groups{}}{\var{groupID}} = (G_e,\, G_t,\, G_p,\, G_c ) \\
        \lookup{\features{}}{\var{parentFeatureID}} = (F_e,\, F_n,\, F_t,\, F_p,\, F_c ) 
      }
      {
        \textbf{addGroup}( \var{groupID}, \var{type}, \var{parentFeatureID} ) \text{ at } \interval{t_n}{t_m} \shove \\
        ( \names{}, \features{}, \groups{} ) \\
        \transition \\
        (\names{}, \\
        \lookup{\features{}}{\var{parentFeatureID}} \assign (F_e,\, F_n,\, F_t,\, F_p ,\, F_c \cup \intervalmapping{t_n}{t_m}{\var{groupID}} ), \\ 
      {\lookup{\groups{}}{\var{groupID}}} \assign 
             \var{setGroupAttributes}( \lookup{\groups{}}{\var{groupID}}, \var{type}, \var{parentFeatureID} )  )
      }
    \end{array}$$
  \caption{\label{rule:add-group}}
\end{figure}

\begin{figure}[ht]
  \begin{minted}[escapeinside=||]{text}
setGroupAttributes(|$(F_e, F_t, F_p, F_c)$|, |$\interval{t_{start}}{t_{end}}$|, type, parentFeatureID)
  = |$($| |$F_e \cup \interval{t_{start}}{t_{end}}$|
    , |$F_n$|[|$\interval{t_{start}}{t_{end}}$|] |$\assign$| name
    , |$F_t$|[|$\interval{t_{start}}{t_{end}}$|] |$\assign$| type
    , |$F_p$|[|$\interval{t_{start}}{t_{end}}$|] |$\assign$| parentFeatureID
    , |$F_c$| |$)$|
     \end{minted}
  \caption{\label{fun:set-group-attributes}}
\end{figure}

\todo{create \var{addChildGroup} function}

\begin{figure}[ht]
  \begin{minted}[escapeinside=||]{text}
addChildGroup|$\left(\left(F_e, F_n, F_t, F_p, F_c \right), \,\interval{t_{start}}{t_{end}}, \, \var{gid}\right)$|
  = |$\left(F_e,\, F_n,\, F_t,\, F_p,\, F_c \interval{t_{start}}{t_{end}} \addassign \var{gid}\right)$|
  \end{minted}
  \caption{\label{fun:add-child-group}}
\end{figure}

